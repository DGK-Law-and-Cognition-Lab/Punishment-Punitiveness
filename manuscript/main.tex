\documentclass[man,12pt,natbib,floatsintext,nolmodern]{apa7}

% === PACKAGES ===
\usepackage[utf8]{inputenc}
\usepackage[T1]{fontenc}
\usepackage{mathptmx}          % Times New Roman (APA 7 / L&HB requirement)
\usepackage{amsmath}
\usepackage{graphicx}
\usepackage{booktabs}
\usepackage{array}
\usepackage{threeparttable}
\usepackage{hyperref}
\usepackage{xcolor}
\usepackage{csquotes}

% === HYPERLINK SETUP ===
\hypersetup{
    colorlinks=true,
    linkcolor=black,
    citecolor=black,
    urlcolor=blue,
    pdfauthor={David G. Kamper, David E. Melnikoff, Dan Simon},
    pdftitle={Is Criminal Punishment Prosocial?}
}

% === GRAPHICS PATH ===
\graphicspath{{figures/}}

% === DOCUMENT METADATA ===
\title{Is Criminal Punishment Prosocial?}

\authorsnames[1,2,3]{David G. Kamper, David E. Melnikoff, Dan Simon}
\authorsaffiliations{
    {Department of Psychology, University of California, Los Angeles},
    {Graduate School of Business, Stanford University},
    {Gould School of Law and Department of Psychology, University of Southern California}
}

\leftheader{Kamper, Melnikoff, \& Simon}
\shorttitle{Is Criminal Punishment Prosocial?}

\abstract{Criminal punishment is said to be justified by its promotion of desirable societal goals: restoring justice, reducing crime, reinforcing societal norms, and rehabilitating offenders. Yet these lofty goals are hard to reconcile with the massive and racialized incarceration regimen that prevails in the United States. We explore this tension by probing popular punitiveness, a major driver of punishment policy. Specifically, we examine whether punitiveness may be influenced by goals that diverge from punishment's purported prosociality. Testing a sample of 496 U.S. respondents, we measured how eighteen psychological constructs interact with punitiveness and with one another. We then extended this analysis by applying natural language processing (NLP) to participants' open-ended sentencing justifications, comparing what people \textit{say} motivates their punishment preferences against their psychological profiles. The correlational findings reveal that punitiveness is a highly complex yet cohesive psychological mindset that correlates strongly with beliefs, emotions, attitudes, personality traits, and ideologies---most of which are orthogonal, if not inimical, to its avowed prosociality. Hostile aggression measures (hatred, revenge, degradation, endorsement of suffering) correlate far more strongly with punitiveness than do legitimate crime concerns. The NLP analyses reveal that participants overwhelmingly invoke prosocial language---citing deterrence, public safety, and rehabilitation---regardless of whether they score high or low on hostile aggression. Across four independent classification methods, 67.5\% of sentencing justifications are semantically closer to revenge and suffering themes than to deterrence and rehabilitation. This prosocial framing functions not as individual-level deception but as a collective cultural default: a shared rhetorical script that obscures the darker psychological drivers of punitiveness. These findings call for a shift in the prevailing discourse to include a frank acknowledgment that criminal punishment serves also to satisfy the punishing public's own---often unsavory---psychological needs.}

\keywords{punitiveness, criminal punishment, prosocial facade, hostile aggression, natural language processing, sentencing, mass incarceration}

% Corresponding author
\authornote{
    \addORCIDlink{David G. Kamper}{} 
    \addORCIDlink{David E. Melnikoff}{}
    \addORCIDlink{Dan Simon}{}
    
    David G. Kamper, Department of Psychology, University of California, Los Angeles. David E. Melnikoff, Graduate School of Business, Stanford University. Dan Simon, Gould School of Law and Department of Psychology, University of Southern California.
    
    This study was preregistered on the Open Science Framework (\url{https://osf.io/kr7y2/}). Data, materials, and analysis code are publicly available at \url{https://dgk-law-and-cognition-lab.github.io/Punishment-Punitiveness/}.
    
    Correspondence concerning this article should be addressed to Dan Simon, USC Gould School of Law, University of Southern California, Los Angeles, CA 90089. E-mail: dsimon@law.usc.edu
}

\begin{document}

\maketitle

% === MAIN TEXT SECTIONS ===
% === INTRODUCTION ===
% No heading needed per APA 7 - text begins directly after abstract

Criminal punishment has always been in tension with liberal democratic values and enlightened human sensibilities \citep{Allen1999}. As Jeremy Bentham (1789) reminds us: ``all punishment is mischief: all punishment in itself is evil'' (ch.~13, para.~2). For centuries, this tension has precipitated the inquiry ``why do we punish?'' \citep{Allen1999, Tonry2011}, an endeavor that has sought to couch punishment in a just framework. Classically, the well-honed justifications of punishment consist of retribution, deterrence, incapacitation, norm expression, and rehabilitation \citep[see][]{Norrie2014, Tonry2011}. These justifications dominate our legal and political discourse. They serve as the gateway to legal education \citep{Ristroph2006}, and they figure prominently in courtroom rhetoric, judicial opinions, legislative debates, and political campaigning. We rely on them also when we speak of bringing offenders to justice, getting them off the street, locking them up, and sending them a message.

This paper does not attempt to examine the philosophical soundness of these justifications, nor to explore how they fit with one another (they do not). Rather, we focus on the core tenet that they are said to serve. The justifications claim great fidelity to the pursuit of desirable and race-neutral societal goals: restoring justice, reducing crime, reinforcing social norms, and rehabilitating people who have broken the law---goals that are quintessentially prosocial.

While grounded in moral thinking \citep{Allen1999, Tonry2011}, the prosocial tenet enjoys empirical backing. Experimental economists and evolutionary psychologists highlight the altruistic nature of punishment. Punishment is said to have enabled our predecessors to contain important collective problems such as free-loading, mate poaching, and acts of violence. Punishing such behaviors has served to facilitate non-kin cooperation and promote societal coexistence \citep{BowlesGintis2011, PriceCosmidesTooby2002, ShinadaYamagishi2007, FehrFischbacher2003, FehrFischbacher2004, FehrGachter2000, FehrGachter2002}.

It must not be overlooked that we are conducting this debate in the shadow of a fifty-year punitive regime that is excessive in its scope \citep{NRC2014, Garland2020}, harshness \citep{Gottschalk2016, JSimon2016}, and disparate racial impact \citep{Armour2020, Alexander2010, Western2006}. The question that animates this exploration is how we get from the avowed prosociality of punishment to mass incarceration, and why we maintain this regime while knowing that it causes more harm than good \citep[see][]{NRC2014, Tonry2018}. The answer, we maintain, cannot be chalked up solely to ignorance or institutional inertia. Rather, it may well lie in the composition and force of the psychological motivations that fuel the punitive mindset.

\subsection{The Punitive Mindset}

The psychology of punitiveness is complex, touching on beliefs, emotions, attitudes, personality traits, worldviews, and ideologies. A considerable body of research has explored the correlates of punitive attitudes, revealing a picture that sits uneasily alongside the prosocial premise.

Punitiveness is shaped, in part, by people's perceptions and fears about crime. Belief in high crime rates is associated with greater punitiveness \citep{Hough2005, ApplegateEtAl2002, SprucerEtAl2015}, even though such beliefs are frequently inflated relative to actual crime trends \citep{QuillianPager2010, Pfaff2017}. Fear of becoming a victim also correlates with punitive attitudes \citep{Hough2005}, though this relationship is more modest than is commonly supposed \citep{Gerber2021, Pickett2019}. These crime-related concerns represent the most intuitively prosocial correlates of punitiveness---people who perceive greater threats might reasonably desire stronger responses.

Emotions, however, tell a more complicated story. Anger and outrage toward offenders are potent predictors of punitive responses \citep{Goldberg1999, Stalans1993, Carlsmith2008}, and emotional reactions can override more deliberative considerations \citep{Lerner2000}. Hatred of criminals is closely associated with punitiveness \citep{DSimon2023}, and people endorse harsher punishments when they experience strong negative affect toward a transgressor \citep{Roseman1994, Weiner1995}. These emotional reactions are not necessarily inimical to prosociality---anger at injustice can motivate constructive action---but they move the motivational story beyond the cool rationality implied by the deterrence and incapacitation frameworks.

Perhaps most troubling for the prosocial account, punitiveness appears to be closely intertwined with endorsement of hostile and aggressive treatment of offenders. Punitive attitudes correlate with acceptance of degrading prison conditions \citep{DSimon2023}, tolerance of prison violence, and the desire to inflict suffering on offenders beyond what is required for any utilitarian purpose. People also seem to derive satisfaction from incidental suffering that befalls rule violators \citep{FincherTetlock2015, GillCerce2021}. Punitiveness is closely intertwined with revenge \citep{Jackson2019, GerberJackson2013, McKeeFeather2008}, which is predicted by the arousal of anger \citep{Dewall2007, Chester2016} and is associated with aggression and violence \citep{DewallChester2021}. Revenge often results in disproportionate punishments that exceed the severity of the transgression \citep{Ho2002}. Critically, experimental research has consistently shown that when people's punishment decisions are examined closely, they are better explained by retribution and revenge than by the utilitarian justifications of deterrence or incapacitation that people tend to endorse verbally \citep{Carlsmith2006, Carlsmith2008, CarlsmithDarley2008, AharoniFridlund2012}.

Punitive attitudes are also deeply embedded in broader personality and ideological orientations. Right-wing authoritarianism (RWA) is among the most robust predictors of punitiveness \citep{Gerber2021, GerberJackson2013, AltmeyerRWA}, and social dominance orientation (SDO) is likewise associated with punitive attitudes \citep{Sidanius2004, Pratto1994}. Violence proneness, racial resentment, and vengefulness have all been linked to harsher punishment preferences \citep{KindaGoldberg2020, Unnever2010a, Unnever2010b}. Taken together, these personality and ideological factors paint a portrait of punitiveness that is deeply interwoven with authoritarian, hierarchical, and retaliatory dispositions.

This partial review reveals that punitiveness is a highly complex and fraught psychological construct. Yet this literature is dispersed across a large number of studies, each of which tests a small number of constructs and thus covers limited slivers of the problem space \citep[for partial exceptions, see][]{KingMaruna2009, GerberJackson2013, Okimoto2012}. Consequently, the field has lacked a macro-level view that gauges the relative strength of the various correlate constructs and examines their interrelationships. Additionally, there is no standard operationalization of the focal punitiveness construct \citep[see][]{Gerber2021}: some studies base it on penological attitudes \citep[e.g.,][]{Cullen1985}, others solicit sentencing decisions in response to vignettes \citep[e.g.,][]{Carlsmith2008}, and yet others tap support for specific sentencing policies \citep[e.g.,][]{TylerBoeckmann1997}.

\subsection{The Problem of Self-Report}

The research reviewed above relies almost exclusively on what people report about themselves on structured questionnaires. This is a problem for studying punitiveness, because the prosocial justifications of punishment are deeply embedded in public discourse. When asked \textit{why} they endorse harsh sentences, people reliably invoke deterrence, public safety, and incapacitation \citep{Carlsmith2008, Roberts2003}. Yet a separate line of experimental research demonstrates that these stated justifications often do not match people's actual punishment behavior: people endorse deterrence in principle but punish according to retributive severity in practice \citep{Carlsmith2006, CarlsmithDarley2008}.

This discrepancy raises a critical question: when people justify their punishment preferences in their own words, are they expressing their genuine motivations, or are they drawing on a culturally available script? If the prosocial language of deterrence and rehabilitation functions as a rhetorical veneer---a \textit{prosocial facade}---then examining people's spontaneous justifications with computational text analysis may reveal the gap between what they say and what drives them.

\subsection{The Present Study}

The present study addresses these limitations with two complementary approaches. First, we conduct a comprehensive correlational investigation of the punitiveness ecosystem, measuring how eighteen psychological constructs---organized into four thematic clusters (crime concerns, emotional reactions, hostile aggression endorsement, and personality/ideological factors)---relate to punitiveness and to one another. This design enables us to examine the field at both a macro-level cluster view and a detailed construct-level view, providing the kind of integrative framework that has been lacking in the literature. We operationalize punitiveness using a multi-method composite that combines penological attitudes, policy preferences, and a sentencing decision across three criminal vignettes, addressing the longstanding operationalization problem.

Second, and centrally, we extend this correlational approach by collecting open-ended sentencing justifications and subjecting them to a multi-method natural language processing (NLP) pipeline. Following their sentencing decision, participants were asked to explain their reasoning in their own words. We then applied four independent computational text analysis methods---dictionary-based coding, zero-shot classification with large language models, forced-choice classification, and Sentence-BERT semantic similarity---to characterize these justifications. This approach allows us to address the prosocial facade hypothesis directly: we can test whether participants' spontaneous language aligns with the prosocial justifications or with the darker psychological factors that correlate with their attitudes.

This study is, to our knowledge, the first to combine a comprehensive psychological profiling of punitiveness with computational text analysis of punishment justifications. The NLP approach offers several advantages over prior methods. Unlike experimental paradigms that pit deterrence against retribution in artificial trade-off scenarios \citep{Carlsmith2006}, text analysis examines the language people naturally produce. Unlike human coding, computational methods are scalable, reproducible, and less susceptible to coder bias. And unlike prior qualitative studies of punishment language, multi-method NLP can quantify the semantic content of justifications and correlate it directly with the psychological profiles established through the quantitative measures.

We preregistered the following primary hypotheses (\url{https://osf.io/kr7y2/}). \textbf{Hypothesis 1} predicted that punitiveness would be positively correlated with all correlate measures. \textbf{Hypothesis 2} predicted that the hostile aggression cluster would show stronger correlations with punitiveness than the crime concerns cluster. \textbf{Hypothesis 3} predicted that most correlate measures would be positively intercorrelated, suggesting a cohesive psychological ``punitiveness mindset.''

For the NLP analyses, we tested two additional predictions. \textbf{Hypothesis 4} predicted that participants would predominantly use prosocial language (deterrence, public safety, rehabilitation) in their sentencing justifications, even when their psychological profiles suggest hostile motivations. \textbf{Hypothesis 5} predicted that the semantic content of justifications would be more closely aligned with revenge and suffering themes than with deterrence and rehabilitation, despite surface-level prosocial framing.

\section{Method}

This study was preregistered on the Open Science Framework prior to data collection (\url{https://osf.io/kr7y2/}). All data, materials, analysis code, and an interactive results website are publicly available at \url{https://dgk-law-and-cognition-lab.github.io/Punishment-Punitiveness/}. We report how we determined our sample size, all data exclusions, all manipulations, and all measures \citep{simmons2012false}. The quantitative analyses were conducted in R \citep{Rcore2024}; the NLP pipeline was implemented in Python.

\subsection{Participants}

A sample of 496 U.S. adults was recruited through Prolific, using Prolific's nationally representative sampling feature, which applies quota balancing on age, sex, and ethnicity. The preregistered target was 480, following the sample size of a preliminary version of the study \citep{SimonMelnikoff2025}. From an initial pool of 538 respondents, we excluded those who did not complete the survey ($n = 8$) and those who failed either of two embedded attention checks ($n = 34$). Participants were compensated at Prolific's standard rate. The study was approved by the Institutional Review Board at the University of Southern California.

The final sample had a mean age of 46.2 years ($SD = 16.2$, range: 19--85). Gender composition was 49.2\% male, 50.2\% female, and 0.6\% other. Racial and ethnic composition was 65.3\% White, 13.3\% Black, 10.7\% Hispanic/Latino, 7.9\% Asian, and 2.8\% other. Political identification was evenly distributed, with 38.3\% liberal, 23.4\% moderate, and 38.3\% conservative.

\subsection{Design and Procedure}

The study employed a correlational design. All participants completed the same battery of measures in a fixed order, with one exception: for the sentencing task, each participant was randomly assigned to one of three criminal vignettes, designed to enhance the generalizability of the findings across crime types. The survey began with punitiveness attitude measures, followed by crime concern measures, hostile aggression and emotion measures, personality and ideology measures, the randomly assigned criminal vignette with a sentencing decision, an open-ended sentencing justification, and demographics.

\subsection{Criminal Vignettes}

Each participant read one of three vignettes describing a second-degree murder case involving a defendant named ``Darryl Smith.'' The vignettes varied the circumstances while holding constant the legal charge and defendant. In Vignette A (Stranger Felony-Murder, $n = 168$), Darryl, recently unemployed, attempts a purse-snatching; the victim resists, a struggle ensues, Darryl strikes her, and she falls over a railing to her death. In Vignette B (Domestic Violence, $n = 176$), Darryl violates a restraining order, confronts his ex-partner, and a physical altercation results in her death. In Vignette C (Organized Crime, $n = 152$), Darryl is involved in a car theft operation and assaults a rival, who dies from the injuries.

After reading the assigned vignette, participants provided a sentencing recommendation on a 0--50 year scale and then responded to the open-ended prompt: ``In your own words, please explain why you recommended this sentence for Darryl. What was your reasoning?'' A follow-up prompt encouraged elaboration, and the two responses were concatenated for analysis.

\subsection{Measures}

All quantitative measures used 7-point Likert scales (1 = \textit{Strongly disagree} to 7 = \textit{Strongly agree}) unless otherwise noted. Reliability estimates for all scales are reported in Table~\ref{tab:reliability}. Full item wordings are available in the supplementary materials.

\subsubsection{Punitiveness}

The primary dependent variable was a standardized composite capturing three distinct operationalizations of punitive attitudes. Penological attitudes were assessed with four items measuring endorsement of more punishment (2 items) and rejection of the parsimony principle (2 items). Policy preferences were assessed with four items tapping support for three-strikes laws (2 items), life without parole (1 item), and the death penalty (1 item). These eight items formed a reliable composite ($\alpha = .84$). The sentencing decision---the number of years assigned to the defendant---was $z$-scored within each vignette condition to account for cross-vignette differences in mean sentence length, and then combined with the $z$-scored 8-item attitude composite to produce the overall punitiveness aggregate.

\subsubsection{Correlate Measures}

The correlate measures were organized into four thematic clusters. The \textit{crime concerns} cluster comprised perceived crime rates (2 items, $\alpha = .85$; e.g., ``Crime in America is at an all-time high'') and fear of crime (3 items, $\alpha = .78$; e.g., ``I am afraid of becoming a victim of a violent crime''). The \textit{emotional reactions} cluster comprised hatred of criminals (3 items, $\alpha = .78$; e.g., ``I hate people who commit violent crimes'') and anger toward criminals (2 items, $\alpha = .83$).

The \textit{hostile aggression} cluster, which is central to this study's hypotheses, comprised six constructs: social exclusion (3 items, $\alpha = .76$; endorsement of civic exclusion of offenders), degradation (3 items, $\alpha = .67$; acceptance of shaming and degrading treatment), infliction of suffering (2 items, $\alpha = .79$; endorsement of suffering as an appropriate consequence), tolerance of prison violence (2 items, $\alpha = .56$), endorsement of harsh prison conditions (3 items, $\alpha = .82$), and revenge (3 items, $\alpha = .67$; endorsement of retaliatory punishment). The composite of all six constructs showed excellent reliability ($\alpha = .92$). Two constructs in this cluster---prison violence tolerance and degradation---had reliabilities below .70 and are interpreted with appropriate caution.

The \textit{personality and ideology} cluster comprised right-wing authoritarianism (RWA; 5 items, $\alpha = .87$; adapted from \citealt{AltmeyerRWA}), social dominance orientation (SDO; 8 items, $\alpha = .92$; adapted from \citealt{Pratto1994}), vengefulness (5 items, $\alpha = .88$; dispositional tendency to seek revenge), violence proneness (4 items, $\alpha = .73$), racial resentment (4 items, $\alpha = .90$; adapted from \citealt{HenrySears2002}), and blood sports viewership (4 items, $\alpha = .84$; frequency of watching boxing, MMA/UFC, wrestling, and hunting, extracted from a broader 9-item media diet matrix).

Two additional constructs were collected for exploratory purposes and are not included in the primary cluster framework: tolerance of due process violations (3 items, $\alpha = .60$) and willingness to convict on uncertain evidence (4 items, $\alpha = .67$).

\subsection{Natural Language Processing Pipeline}

We applied four independent computational methods to classify and quantify the thematic content of participants' open-ended sentencing justifications. The concatenated responses had a mean word count of $M = 48.3$ ($SD = 35.9$).

\subsubsection{Classification Methods}

The first method used custom dictionaries constructed for ten justification categories spanning prosocial themes (deterrence, incapacitation, rehabilitation, retribution, norm expression) and dark themes (revenge, suffering, degradation, exclusion, victim focus). Each response was scored by the proportion of words matching each dictionary and classified by its highest-scoring category.

The second method employed BART-large-MNLI \citep{lewis2020bart} for zero-shot multi-label classification against eight candidate labels: proportional justice, victim closure, rehabilitation and reform, deterrence and prevention, public safety and protection, and societal condemnation (prosocial), as well as punishment and suffering, and revenge and payback (dark). The third method used the same model in a forced-choice format, assigning each response to a single best-fitting category.

The fourth method computed Sentence-BERT \citep{reimers2019sentence} cosine similarity between each response and prototype embeddings representing seven justification themes. Prototype sentences were constructed to represent canonical expressions of each justification (e.g., ``This sentence will deter others from committing similar crimes'' for deterrence; ``He deserves to suffer for what he did'' for suffering). Prosocial and dark similarity composites were computed as the mean of the respective category similarities.

\subsubsection{Validation and Additional Measures}

We validated the pipeline in three ways: benchmark accuracy was assessed using 40 hand-crafted sentences with known ground-truth labels; cross-method convergence was quantified through pairwise agreement and Cohen's $\kappa$ across all four methods; and a prototype sensitivity analysis re-estimated the dark-closer percentage using two alternative sets of prototype sentences (formal academic framings and colloquial framings) to ensure robustness to prototype wording.

We also computed VADER sentiment scores \citep{hutto2014vader} for each response, scored responses on 25 thematic categories using the Empath lexicon \citep{fast2016empath}, and applied BERTopic \citep{grootendorst2022bertopic} to discover emergent thematic clusters in the corpus.

\subsection{Analytic Strategy}

The correlational analyses used Pearson correlations between punitiveness and all correlate constructs. To test whether hostile aggression outpredicts crime concerns (H2), we applied Steiger's $Z$-test for comparing dependent correlations \citep{steiger1980tests} and computed bootstrapped 95\% confidence intervals (10,000 iterations). False discovery rate correction was applied for construct-level tests.

The NLP analyses proceeded in three stages. First, we characterized the distribution of justification themes across the four classification methods. Second, we tested the individual-level facade hypothesis by correlating text features with hostile aggression, crime concerns, and punitiveness. Third, we adjudicated among three competing interpretations of any observed mismatch between prosocial language and psychological profiles: individual facade (deception), sincerity (genuine crime concerns), and cultural default (shared rhetorical script).

\section{Results}

\subsection{Preliminary Analyses}

\subsubsection{Sample Characteristics}

The final sample comprised 496 U.S. adults. Participants were randomly assigned to one of three vignettes: Stranger Felony-Murder ($n = 168$), Domestic Violence ($n = 176$), and Organized Crime ($n = 152$). Political orientation was assessed on a 7-point scale and categorized into Liberal ($n = 190$), Moderate ($n = 116$), and Conservative ($n = 190$).

\subsubsection{Descriptive Statistics}

The punitiveness composite ($M = 0.00$, $SD = 0.82$) was centered at zero by construction. Among the correlate measures, the highest means were observed for anger toward criminals ($M = 4.84$, $SD = 1.51$), perceived crime rates ($M = 5.00$, $SD = 1.42$), and exclusion ($M = 4.46$, $SD = 1.45$). The lowest means were for social dominance orientation ($M = 2.58$, $SD = 1.44$), vengefulness ($M = 2.97$, $SD = 1.41$), and fear of crime ($M = 3.02$, $SD = 1.34$). Notably, the hostile aggression composite ($M = 3.77$, $SD = 1.21$) fell near the scale midpoint, with 42\% of participants scoring above the midpoint. Full descriptive statistics are reported in Table~\ref{tab:descriptives}.

\subsubsection{Reliability}

All primary composites demonstrated acceptable to excellent internal consistency. The punitiveness composite ($\alpha = .84$), the hostile aggression cluster ($\alpha = .92$), and the personality cluster ($\alpha = .92$) all exceeded $\alpha = .80$. Two individual constructs---parsimony ($\alpha = .48$) and prison violence tolerance ($\alpha = .56$)---fell below conventional thresholds and are interpreted with caution. Full reliability estimates are reported in Table~\ref{tab:reliability}.

\subsection{Hypothesis 1: Punitiveness Correlates With All Measures}

Hypothesis 1 predicted that punitiveness would be positively correlated with all correlate measures. This prediction was strongly supported: 16 of 18 constructs were significantly and positively correlated with punitiveness (all $p < .01$ after FDR correction). The two exceptions were vengefulness ($r = .12$, $p = .074$) and blood sports viewership ($r = .11$, $p = .080$), which were positive but fell short of significance. The full results are presented in Table~\ref{tab:H1correlations} and Figure~\ref{fig:heatmap_construct}.

The strongest correlates of punitiveness were: endorsement of harsh prison conditions ($r = .56$), social exclusion ($r = .54$), hatred of criminals ($r = .53$), infliction of suffering ($r = .52$), right-wing authoritarianism ($r = .50$), and racial resentment ($r = .46$). By contrast, the crime concern measures---perceived crime rates ($r = .40$) and fear of crime ($r = .18$)---correlated substantially less strongly with punitiveness. The pattern is clear: measures tapping hostile, aggressive, and ideological dispositions correlate with punitiveness at least as strongly as---and typically more strongly than---measures of legitimate concern about crime.

\subsection{Hypothesis 2: Hostile Aggression Outpredicts Crime Concerns}

Hypothesis 2 predicted that the hostile aggression cluster would correlate more strongly with punitiveness than the crime concerns cluster. This was tested using Steiger's $Z$-test for dependent correlations \citep{steiger1980tests}. The hostile aggression composite correlated with punitiveness at $r = .59$, whereas the crime concerns composite correlated at $r = .32$. This difference was highly significant ($Z = 6.26$, $p < .001$).\footnote{When punitiveness is measured using the 8-item attitude composite alone (excluding the sentencing decision), the hostile aggression correlation increases to $r = .71$ ($r_{\text{crime}} = .36$), producing an even larger difference.}

We further tested this hypothesis for each individual punitiveness measure to ensure robustness. As shown in Table~\ref{tab:steiger}, hostile aggression significantly outpredicted crime concerns for five of six punitiveness measures: Punish More ($\Delta r = .34$, $Z = 8.23$, $p < .001$), Parsimony ($\Delta r = .31$, $Z = 6.70$, $p < .001$), Death Penalty ($\Delta r = .32$, $Z = 7.23$, $p < .001$), Three Strikes ($\Delta r = .22$, $Z = 5.06$, $p < .001$), and LWOP ($\Delta r = .15$, $Z = 3.27$, $p = .001$). The one exception was the sentencing decision ($\Delta r = .08$, $Z = 1.60$, $p = .11$), for which the difference was in the predicted direction but not statistically significant.

Bootstrapped 95\% confidence intervals (10,000 iterations) further confirmed that the advantage of hostile aggression over crime concerns was robust (Figure~\ref{fig:bootstrap}). At the cluster level, the observed difference of $r = .264$ had a bootstrap CI of $[.174, .367]$, excluding zero. The same held for the emotions cluster (difference $= .213$, CI $[.115, .321]$) and the personality cluster (difference $= .153$, CI $[.057, .259]$).

\subsubsection{Tautology Sensitivity Analysis}

A potential concern is that some hostile aggression constructs---particularly suffering, harsh conditions, prison violence, and degradation---may overlap conceptually with punitiveness, inflating the hostile aggression correlation through construct overlap rather than genuine predictive superiority. We addressed this through a series of sensitivity analyses in which we progressively removed the most tautology-prone constructs from the hostile aggression composite. In the conservative test (retaining only exclusion, degradation, prison violence, and revenge), the hostile aggression advantage held: $r = .55$ vs.\ $r = .32$, $Z = 5.07$, $p < .001$. In the strictest test (retaining only exclusion and revenge---the two constructs with no plausible item overlap with punitiveness), the advantage remained: $r = .54$ vs.\ $r = .32$, $Z = 5.05$, $p < .001$. The strictest test held for five of six individual punitiveness measures (all $Z > 2.91$, $p < .004$; the exception was again the sentencing decision). We further tested robustness to the lowest-reliability component of the punitiveness composite by dropping parsimony ($\alpha = .48$); results were virtually unchanged ($r = .58$ vs.\ $r = .33$, $Z = 5.87$, $p < .001$). Even in the most conservative specification---only two hostile aggression constructs and no parsimony items---the hostile aggression advantage was significant ($r = .55$ vs.\ $r = .33$, $Z = 5.02$, $p < .001$). Full details are reported in the Supplementary Materials.

\subsection{Hypothesis 3: Cohesive Punitiveness Mindset}

Hypothesis 3 predicted that the correlate measures would be positively intercorrelated, forming a cohesive psychological mindset. This prediction was strongly supported. All 153 pairwise correlations among the 18 constructs were positive, ranging from $r = .02$ (fear--racial resentment) to $r = .72$ (hatred--anger), and the vast majority were statistically significant. The full intercorrelation matrix is presented in the supplementary materials (Table~S1).

Several notable patterns emerged. The hostile aggression constructs formed a tightly interrelated cluster, with intercorrelations ranging from $r = .38$ (prison violence--revenge) to $r = .71$ (exclusion--degradation). Emotional reactions (hatred and anger) were strongly intercorrelated ($r = .72$) and showed robust associations with hostile aggression constructs ($r$s = .15--.64, with the lower values involving prison violence tolerance, the least reliable construct in the cluster). Right-wing authoritarianism and social dominance orientation, while moderately correlated with each other ($r = .43$), showed differential patterns: RWA correlated more strongly with crime-related constructs (crime rates: $r = .44$; anger: $r = .49$) than did SDO ($r$s = .20 and .24, respectively), whereas SDO correlated more strongly with racial resentment ($r = .62$ vs.\ $r = .53$).

The two constructs that fell outside the mindset---vengefulness and blood sports---provide an informative contrast. Vengefulness was weakly correlated with punitiveness ($r = .12$) but moderately correlated with revenge ($r = .46$), degradation ($r = .37$), and suffering ($r = .40$). This suggests that \textit{dispositional} vengefulness (the tendency to seek revenge in everyday life) is dissociable from the \textit{domain-specific} endorsement of harsh criminal punishment, even though both share variance with aggressive treatment of offenders. Blood sports viewership was weakly associated with the entire mindset, suggesting that preference for violent entertainment operates largely independently of the punitiveness ecosystem.

\subsection{Vignette Stability}

Because participants were randomly assigned to one of three vignettes, we examined whether the correlational patterns were stable across crime types. The punitiveness--correlate correlations were highly stable: across the 18 constructs, the range of correlations across vignettes was small (median range $= .14$), and no construct reversed direction across conditions. The hostile aggression advantage over crime concerns was replicated within each vignette condition. This stability indicates that the punitiveness mindset is not an artifact of the particular crime scenario presented to participants.

\subsection{Natural Language Processing Results}

The NLP analyses pursue two questions. First, is there a gap between the prosocial language people use to justify punishment and the semantic content of what they write? Second, if such a gap exists, what explains it---individual deception, genuine sincerity, or something else?

\subsubsection{The Prosocial Surface}

Across the 496 open-ended justifications, participants overwhelmingly invoked prosocial themes. When zero-shot classification assigned each response to its single most probable category, deterrence and prevention dominated (50.6\%), followed by proportional justice (11.9\%), public safety (9.1\%), and societal condemnation (7.5\%). Punishment and suffering was the top category in only 7.9\% of responses, and revenge and payback in 4.6\%. Multi-label classification told a similar story: the mean endorsement probability was highest for deterrence ($M = .85$), followed by proportional justice ($M = .80$), societal condemnation ($M = .73$), and public safety ($M = .65$). Collapsing across prosocial and dark categories, 87.5\% of responses were classified as primarily prosocial by zero-shot top-label assignment, 80.0\% by forced-choice classification, and 83.1\% by dictionary-based coding.

At face value, these results suggest that people justify punishment in prosocial terms. But a closer examination reveals a more complicated picture.

\subsubsection{The Semantic Mismatch}

Although participants framed their justifications in prosocial language, the \textit{semantic content} of those justifications was more closely aligned with dark themes than with prosocial ones. Using Sentence-BERT embeddings, we computed each response's cosine similarity to prototypical prosocial statements (deterrence, rehabilitation, incapacitation, retribution) and prototypical dark statements (revenge, suffering, exclusion). Despite the prosocial framing, 67.5\% of responses were semantically \textit{closer} to the dark prototypes than to the prosocial ones ($M_{\text{prosocial}} = .39$, $M_{\text{dark}} = .42$, paired $t(495) = 9.63$, $p < .001$, $d = 0.43$).

This finding was robust across multiple specifications. Using formal academic prototype sentences, 73.0\% of responses were closer to dark prototypes; using colloquial everyday prototypes, 89.7\% were. To address the concern that retribution could be classified as dark rather than prosocial, we tested two alternative specifications: removing retribution from the prosocial set (yielding a balanced 3 vs.\ 3 comparison) and moving retribution to the dark set. In both cases the finding strengthened---78.6\% and 83.5\% of responses were closer to dark prototypes, respectively---because retribution had been helping the prosocial side. The four NLP methods converged on the central pattern: prosocial categories dominated surface-level classifications (80--88\%), but the deeper semantic content tilted toward dark themes.\footnote{The two BART-based methods showed strong agreement (90.5\%, $\kappa = .66$). Agreement between BART and the dictionary or semantic similarity methods was more modest ($\kappa = .03$--.11), as expected given their different operating principles. Details are reported in the Supplementary Materials.}

\subsubsection{Three Competing Explanations}

What explains the mismatch between prosocial framing and dark semantic content? We tested three hypotheses, each making a distinct prediction about how the prosocial--dark gap should relate to individual differences.

\textit{Individual facade.} High-hostile individuals strategically adopt prosocial language to mask their darker motivations. This predicts that the prosocial--dark gap should correlate \textit{negatively} with hostile aggression: the more hostile the person, the more their prosocial framing should diverge from their dark content.

\textit{Sincerity.} Prosocial language reflects genuine crime concerns; the mismatch occurs because even crime-concerned individuals happen to produce language that resembles dark prototypes. This predicts that the prosocial--dark gap should correlate \textit{positively} with crime concerns.

\textit{Cultural default.} Everyone draws on the same culturally available prosocial script regardless of their psychological profile. The mismatch is a collective phenomenon, not an individual one. This predicts that the gap should be uncorrelated with \textit{both} hostile aggression and crime concerns.

The data decisively favored the cultural default. As shown in Table~\ref{tab:nlp_facade} and Figure~\ref{fig:facade_scatter}, hostile aggression was uncorrelated with prosocial language similarity ($r = -.04$, $p = .35$), dark language similarity ($r = -.01$, $p = .83$), the prosocial--dark gap ($r = -.03$, $p = .45$), and sentiment ($r = -.05$, $p = .25$). Two One-Sided Tests (TOST) equivalence testing with bounds of $\pm .15$ confirmed that four of five facade-relevant correlations were formally equivalent to zero (all $p_{\text{TOST}} \leq .014$); the one exception was dictionary-coded prosocial content ($r = -.08$, $p_{\text{TOST}} = .055$). Crime concerns were equally uninformative: the correlation between crime concerns and the prosocial--dark gap was $r = .06$ ($p = .20$). The facade and sincerity hypotheses were both unsupported.

Group comparisons reinforced this conclusion. Participants in the top and bottom tertiles of hostile aggression produced virtually identical language profiles (prosocial similarity: $M_{\text{low}} = .39$ vs.\ $M_{\text{high}} = .39$; dark similarity: $M_{\text{low}} = .42$ vs.\ $M_{\text{high}} = .42$; all $p$s $> .10$, all $d$s $< .16$). These null results were robust to the retribution classification: hostile aggression remained uncorrelated with both prosocial similarity and the prosocial--dark gap under all three prototype specifications (all $p$s $> .37$). Parallel analyses using dictionary prosocial scores, zero-shot category probabilities, and Empath categories yielded the same pattern, with the exception of a marginal trend for dictionary-coded prosocial content ($r = -.08$, $p = .08$).

The prosocial--dark gap was also uncorrelated with hatred ($r = .04$, $p = .41$), SDO ($r = -.06$, $p = .20$), RWA ($r = .04$, $p = .38$), and racial resentment ($r = -.06$, $p = .17$). Only revenge ($r = -.09$, $p = .04$) and punitiveness itself ($r = -.17$, $p < .001$) showed even marginal relationships. The cultural default interpretation treats the prosocial framing as a shared rhetorical resource---a folk theory of punishment that people absorb from legal discourse and media, rather than a deliberate strategy of concealment.

\subsubsection{The Cultural Default Across the Political Spectrum}

If the prosocial framing is a cultural default, it should operate similarly for liberals and conservatives despite their divergent levels of punitiveness. As shown in Table~\ref{tab:political_moderation}, conservatives were substantially more punitive ($M = 0.26$) and more hostile ($M = 4.12$) than liberals ($M = -0.31$ and $M = 3.33$, respectively). Yet their language was virtually identical: prosocial similarity was $.38$ for conservatives and $.39$ for liberals; dark similarity was $.41$ and $.43$; and the percentage of responses closer to dark prototypes was 64.7\% for conservatives and 67.9\% for liberals.

Formal interaction tests confirmed these patterns. The interaction between hostile aggression and political orientation (continuous) predicting prosocial similarity was nonsignificant ($b = -.002$, $p = .26$), as were interactions predicting the prosocial--dark gap ($b = -.002$, $p = .24$) and sentiment ($b = .004$, $p = .70$). Everyone used prosocial language to justify punishment regardless of their psychological profiles, and political orientation did not moderate this pattern.

\subsubsection{Text Features as Independent Predictors}

Although text features were unrelated to hostile aggression, they were related to punitiveness itself. More punitive individuals used less prosocial language ($r = -.13$, $p = .005$), had a more negative prosocial--dark gap ($r = -.17$, $p < .001$), and expressed more negative sentiment ($r = -.19$, $p < .001$). These correlations, while modest, indicate that punitiveness is weakly reflected in language---but this signal is far too weak to differentiate individuals by their hostile aggression.

To assess whether text features contribute unique variance beyond the quantitative measures, we conducted hierarchical regression (with vignette as a fixed-effects covariate). Regressing punitiveness on the four psychological clusters yielded $R^2 = .387$. Adding prosocial similarity, dark similarity, and sentiment increased this to $R^2 = .443$ ($\Delta R^2 = .055$, $F(3, 486) = 16.0$, $p < .001$). Prosocial similarity was the second-largest predictor ($\beta = -.23$, $p < .001$), behind only hostile aggression ($\beta = .32$, $p < .001$). Text features thus capture an aspect of punitiveness not fully explained by the psychological batteries---even though text features are themselves unrelated to hostile aggression.

Text features were even more informative for the sentencing decision itself. Rehabilitation language was the strongest predictor of shorter sentences ($r = -.45$, $p < .001$), followed by sentiment ($r = -.22$); revenge language predicted longer sentences ($r = .29$, $p < .001$). In hierarchical regression, psychological measures alone predicted sentencing at $R^2 = .117$; adding rehabilitation language and sentiment increased this to $R^2 = .258$ ($\Delta R^2 = .141$, $F(2, 488) = 46.5$, $p < .001$), and hostile aggression dropped to nonsignificance ($b = -0.01$, $p = .80$). The text features appear to capture the pathway through which dispositions translate into decisions: people who invoke rehabilitation give shorter sentences, and people who invoke revenge give longer ones, regardless of their hostile aggression scores.

\section{Discussion}

This study set out to probe the prosocial premise of criminal punishment---the bedrock assumption that punishment serves desirable societal goals such as deterrence, public safety, and rehabilitation. We pursued this question through two complementary lenses: a comprehensive correlational mapping of the psychological factors associated with punitiveness, and a multi-method computational text analysis of the language people use to justify their punishment preferences. Together, these approaches paint a picture that is difficult to reconcile with the prosocial narrative.

\subsection{The Punitiveness Mindset}

The correlational findings reveal that punitiveness is embedded in a dense, cohesive psychological ecosystem. Virtually all of the constructs we measured---spanning beliefs, emotions, aggressive dispositions, personality traits, and political ideologies---correlated positively with punitiveness and with one another. This is not a scattershot collection of weak associations. The hostile aggression cluster---comprising hatred of criminals, endorsement of degradation, desire for suffering, tolerance of prison violence, support for harsh conditions, and revenge---correlated with punitiveness at $r = .59$, dramatically exceeding the $r = .32$ correlation observed for crime concerns. This pattern held across five of six individual punitiveness measures, was confirmed by bootstrapped confidence intervals, and survived a series of tautology sensitivity tests that progressively stripped the hostile aggression cluster down to its two most conceptually distinct components (exclusion and revenge; $r = .54$ vs.\ $r = .32$, $Z = 5.05$, $p < .001$).

These findings converge with and extend prior research linking punitiveness to authoritarian dispositions \citep{GerberJackson2013}, negative emotions \citep{Goldberg1999}, and revenge motives \citep{Ho2002, Jackson2019}. What the present study adds is the \textit{relative magnitude} of these associations: when hostile aggression, crime concerns, emotional reactions, and personality factors are examined side by side, it is the aggressive, retaliatory, and hierarchical dispositions that show the strongest ties to punitive attitudes---not the concern about crime that would be expected under the prosocial account.

\subsection{What People Say vs.\ What Drives Them}

The NLP findings introduce a new dimension to this picture. When asked to explain their sentencing decisions in their own words, participants overwhelmingly invoked prosocial language. Deterrence, public safety, and proportional justice were the dominant themes across all classification methods. At face value, this would seem to affirm the prosocial premise.

But the semantic analysis told a different story. Despite the prosocial framing, 67.5\% of justifications were semantically closer to revenge, suffering, and exclusion themes than to deterrence, rehabilitation, and incapacitation---a finding robust across four independent classification methods, three alternative sets of semantic prototypes, and alternative classifications of the philosophically ambiguous category of retribution. People said ``deterrence'' and ``public safety,'' but the substance of what they wrote resonated more with revenge and suffering.

\subsection{The Prosocial Facade as Cultural Default}

The most important finding is the nature of this mismatch. We hypothesized that it might reflect an individual-level facade: that people high in hostile aggression would strategically adopt prosocial language to mask their darker motivations. This hypothesis was not supported. Hostile aggression was completely uncorrelated with all text features---prosocial language, dark language, the prosocial--dark gap, and sentiment. People high and low in hostile aggression produced virtually identical justifications.

Nor did the mismatch reflect genuine prosocial sincerity: crime concerns were equally uncorrelated with the prosocial--dark gap. Instead, the evidence converged on what we term a \textit{cultural default}---a shared, collectively maintained rhetorical script that people draw on when justifying punishment, regardless of their actual psychological profile. The prosocial language of deterrence and public safety is so deeply embedded in legal discourse, political rhetoric, and media coverage that it functions as the default vocabulary for discussing punishment. Everyone speaks the language of prosociality, but underneath, the motivational substrate varies dramatically.

The political ideology analysis reinforced this interpretation. Conservatives were substantially more punitive and more hostile than liberals, yet their sentencing justifications were linguistically indistinguishable. The cultural default held equally across the political spectrum: liberals, moderates, and conservatives all drew on the same prosocial script, and political orientation did not moderate the relationship between psychological profiles and language use.

\subsection{Implications}

These findings have several implications for how we think about criminal punishment.

First, they challenge the taken-for-granted prosocial framing of punishment. The justifications that dominate our legal and political discourse---deterrence, incapacitation, public safety---may not describe the actual psychological forces that drive punishment policy. The fact that hostile aggression, not crime concern, is the dominant correlate of punitiveness suggests that much of what fuels public demand for harsh sentences has little to do with making society safer and much to do with satisfying retaliatory impulses. The fact that these impulses are dressed in prosocial language makes them harder to identify, scrutinize, and resist.

Second, the cultural default finding suggests that the problem is not one of individual deception. People are not consciously lying when they cite deterrence as their rationale. Rather, they are drawing on a culturally available explanatory framework that has become so pervasive that it functions as the natural way to talk about punishment. This has parallels to what \citet{Wilson2004} has described as the ``adaptive unconscious''---the gap between the reasons we give for our behavior and the actual causes of that behavior. In the domain of punishment, the prosocial justifications may function as what \citet{Haidt2001} calls post-hoc rationalizations: morally satisfying accounts that are constructed after the punitive judgment has already been made on more visceral grounds.

Third, the incremental validity findings suggest that text analysis captures something about punitiveness that standard psychological batteries miss. Open-ended language predicted punitiveness above and beyond hostile aggression, emotions, personality, and crime concerns ($\Delta R^2 = .055$), and it was an even stronger predictor of actual sentencing decisions ($\Delta R^2 = .141$). This points to the potential of computational text analysis as a complementary tool in the study of legal attitudes, capable of tapping dimensions of reasoning that structured scales cannot reach.

Fourth, these findings are relevant to the ongoing debate about mass incarceration in the United States. If punitive policy is driven in substantial part by hostile aggression rather than legitimate crime concerns, and if this fact is obscured by a prosocial rhetorical facade, then the usual policy debates---which take the prosocial goals at face value and argue about whether punishment ``works'' as deterrence or incapacitation---are missing a significant part of the picture. A more honest discourse would acknowledge the role of retaliatory and aggressive motivations and subject them to the same scrutiny we apply to the official justifications \citep[cf.][]{NRC2014, Tonry2018, Garland2020}.

\subsection{Limitations and Future Directions}

Several limitations should be noted. First, the study is cross-sectional and correlational; we cannot establish causal relationships between psychological dispositions and punitiveness. The strong intercorrelations among constructs suggest a common underlying factor, but the causal architecture of this mindset remains to be determined through experimental or longitudinal designs.

Second, the theoretical clustering of constructs into hostile aggression, crime concerns, emotions, and personality was guided by face validity and prior literature rather than confirmed by the data's factor structure. Confirmatory factor analysis of our four-cluster model yielded fit indices below conventional thresholds (CFI $= .86$, RMSEA $= .11$), indicating that the clusters should be understood as conceptual organizers rather than empirically validated latent factors. Relatedly, some hostile aggression constructs (particularly suffering, harsh conditions, and prison violence) share conceptual territory with punitiveness itself, raising a tautology concern. However, sensitivity analyses demonstrated that the hostile aggression advantage over crime concerns survived even when the cluster was stripped to its two most conceptually distinct constructs---exclusion and revenge ($r = .54$ vs.\ $r = .32$, $Z = 5.05$, $p < .001$)---and when parsimony items were simultaneously removed from the punitiveness composite. The substantive conclusion does not depend on the potentially tautological elements.

Third, our sample was recruited through Prolific using nationally representative quotas on age, sex, and ethnicity. Although this approach yields samples that approximate U.S. census demographics more closely than convenience sampling, online samples may still differ from the broader population in ways not captured by quota variables \citep{ChandlerShapiro2016}. The generalizability of our findings to other national contexts or to populations reached through probability sampling requires further investigation.

Fourth, the NLP methods, while multi-method and validated, are imperfect. The semantic similarity approach depends on the prototype sentences used; although sensitivity analyses showed robustness across three prototype sets and alternative classifications of retribution (which strengthened rather than weakened the finding), the specific quantitative estimates (e.g., 67.5\% dark-closer) should be interpreted with appropriate caution. Future research might employ additional text analysis methods, including fine-tuned classification models trained on domain-specific labeled data.

Fifth, our open-ended prompt asked participants to explain their sentencing decision, which may have encouraged post-hoc rationalization. Future studies could elicit justifications before the sentencing decision, or use think-aloud protocols, to examine whether the cultural default operates even when participants are not constructing retrospective explanations.

Sixth, we measured punitiveness in the context of three criminal vignettes involving second-degree murder. The generalizability to other crime types, other legal contexts, or real-world sentencing decisions remains an open question, though the stability of our findings across the three vignettes is encouraging.

\subsection{Conclusion}

Is criminal punishment prosocial? Our findings suggest that the answer is far more complicated than the official discourse implies. The psychological forces most strongly associated with punitiveness---hatred, revenge, degradation, suffering, authoritarianism---are orthogonal to the prosocial goals of deterrence, public safety, and rehabilitation. When people explain their punishment preferences in their own words, they reliably invoke those prosocial goals. But computational analysis of their language reveals that the substance of what they say is more aligned with revenge and suffering than with deterrence and reform. This prosocial framing is not individual-level deception; it is a collective cultural default, a shared rhetorical resource that obscures the darker motivational substrate of punishment.

Coming to terms with this complexity might lead us to reconsider how we think about, talk about, and practice criminal punishment. At a minimum, these findings call for a discourse that demystifies the prevailing prosocial rhetoric and introduces a frank recognition that we punish, in part, to satisfy our own---often unsavory---psychological needs.


% === TABLES AND FIGURES ===
% Note: In APA man mode with floatsintext, these appear inline.
% For submission, tables/figures can also be placed at end.

\begin{table}[htbp]
\caption{Correlations Between Punitiveness and All Correlate Constructs (H1)}
\label{tab:H1correlations}
\begin{threeparttable}
\small
\begin{tabular}{lcccc}
\toprule
Construct & $r$ & $p$ & $p_{\text{FDR}}$ & Sig \\
\midrule
Harsh Conditions    & .56 & $< .001$ & $< .001$ & *** \\
Exclusion           & .54 & $< .001$ & $< .001$ & *** \\
Hatred              & .53 & $< .001$ & $< .001$ & *** \\
Suffering           & .52 & $< .001$ & $< .001$ & *** \\
RWA                 & .50 & $< .001$ & $< .001$ & *** \\
Racial Resentment   & .46 & $< .001$ & $< .001$ & *** \\
Anger               & .45 & $< .001$ & $< .001$ & *** \\
Degradation         & .44 & $< .001$ & $< .001$ & *** \\
Due Process         & .43 & $< .001$ & $< .001$ & *** \\
Revenge             & .43 & $< .001$ & $< .001$ & *** \\
Violence Proneness  & .41 & $< .001$ & $< .001$ & *** \\
Crime Rates         & .40 & $< .001$ & $< .001$ & *** \\
Uncertain Evidence  & .35 & $< .001$ & $< .001$ & *** \\
Prison Violence     & .35 & $< .001$ & $< .001$ & *** \\
SDO                 & .32 & $< .001$ & $< .001$ & *** \\
Fear                & .18 & .001     & .002     & **  \\
Vengefulness        & .12 & .074     & .078     &     \\
Blood Sports        & .11 & .080     & .080     &     \\
\bottomrule
\end{tabular}
\begin{tablenotes}[flushleft]
\small
\item \textit{Note.} $N = 496$. All correlations are Pearson $r$. $p_{\text{FDR}}$ = false discovery rate corrected $p$-values. *** $p < .001$, ** $p < .01$.
\end{tablenotes}
\end{threeparttable}
\end{table}

\begin{table}[htbp]
\caption{Steiger's $Z$-Tests Comparing Hostile Aggression vs.\ Crime Concerns Correlations With Each Punitiveness Measure (H2)}
\label{tab:steiger}
\begin{threeparttable}
\small
\begin{tabular}{lcccccc}
\toprule
Measure & $r_{\text{Hostile}}$ & $r_{\text{Crime}}$ & $\Delta r$ & $Z$ & $p$ & Supported \\
\midrule
Punish More     & .62 & .28 & .34 & 8.23 & $< .001$ & Yes \\
Death Penalty   & .56 & .24 & .32 & 7.23 & $< .001$ & Yes \\
Parsimony       & .49 & .18 & .31 & 6.70 & $< .001$ & Yes \\
Three Strikes   & .55 & .33 & .22 & 5.06 & $< .001$ & Yes \\
LWOP            & .47 & .33 & .15 & 3.27 & .001     & Yes \\
Sentence        & .25 & .17 & .08 & 1.60 & .111     & No  \\
\bottomrule
\end{tabular}
\begin{tablenotes}[flushleft]
\small
\item \textit{Note.} $N = 496$. $r_{\text{Hostile}}$ = correlation with hostile aggression composite; $r_{\text{Crime}}$ = correlation with crime concerns composite. Steiger's $Z$-test for dependent correlations \citep{steiger1980tests}. ``Supported'' = hostile aggression significantly greater than crime concerns at $p < .05$.
\end{tablenotes}
\end{threeparttable}
\end{table}

\begin{figure}[htbp]
\centering
\includegraphics[width=\textwidth]{heatmap_construct_level.png}
\caption{Construct-level correlation heatmap. Pearson correlations between punitiveness and all 18 correlate constructs. Warmer colors indicate stronger positive correlations. The hostile aggression constructs (exclusion through revenge) show the strongest associations with punitiveness.}
\label{fig:heatmap_construct}
\end{figure}

\begin{figure}[htbp]
\centering
\includegraphics[width=.75\textwidth]{bootstrap_h2_distribution.png}
\caption{Bootstrap distribution of the difference in correlations between hostile aggression and crime concerns with punitiveness ($\Delta r = r_{\text{Hostile}} - r_{\text{Crime}}$). The solid red line indicates the observed difference; dashed green lines indicate the 95\% bias-corrected and accelerated confidence interval (10,000 iterations). The entire distribution falls well above zero (dashed black line), confirming that hostile aggression is a substantially stronger correlate of punitiveness than crime concerns.}
\label{fig:bootstrap}
\end{figure}

\begin{figure}[htbp]
\centering
\includegraphics[width=\textwidth]{facade_scatter_plot.png}
\caption{Hostile aggression (quantitative) plotted against prosocial language similarity (Sentence-BERT). Each point represents one participant, colored by punitiveness. The flat regression line ($r = -.04$, $p = .35$) demonstrates that hostile aggression is unrelated to prosocial language use: participants who score high on hostile aggression invoke deterrence, public safety, and rehabilitation just as readily as those who score low. This null relationship is the key evidence against the individual-level prosocial facade hypothesis and for the cultural default interpretation.}
\label{fig:facade_scatter}
\end{figure}

\begin{table}[htbp]
\caption{Reliability Estimates (Cronbach's Alpha) for All Composite Measures}
\label{tab:reliability}
\begin{threeparttable}
\scriptsize
\begin{tabular}{lcc}
\toprule
Construct & $\alpha$ & Items \\
\midrule
\multicolumn{3}{l}{\textit{Punitiveness}} \\
\quad Punish More & .68 & 2 \\
\quad Parsimony & .48 & 2 \\
\quad Three Strikes & .75 & 2 \\
\quad Punitiveness (8-item) & .84 & 8 \\
\midrule
\multicolumn{3}{l}{\textit{Crime Concerns}} \\
\quad Crime Rates & .85 & 2 \\
\quad Fear of Crime & .78 & 3 \\
\midrule
\multicolumn{3}{l}{\textit{Emotions}} \\
\quad Hatred & .78 & 3 \\
\quad Anger & .83 & 2 \\
\midrule
\multicolumn{3}{l}{\textit{Hostile Aggression}} \\
\quad Exclusion & .76 & 3 \\
\quad Degradation & .67 & 3 \\
\quad Suffering & .79 & 2 \\
\quad Prison Violence & .56 & 2 \\
\quad Harsh Conditions & .82 & 3 \\
\quad Revenge & .67 & 3 \\
\midrule
\multicolumn{3}{l}{\textit{Personality/Ideology}} \\
\quad RWA & .87 & 5 \\
\quad SDO & .92 & 8 \\
\quad Vengefulness & .88 & 5 \\
\quad Violence Proneness & .73 & 4 \\
\quad Racial Resentment & .90 & 4 \\
\quad Blood Sports & .84 & 4 \\
\bottomrule
\end{tabular}
\begin{tablenotes}[flushleft]
\small
\item \textit{Note.} $N = 496$. Hostile aggression cluster composite ($\alpha = .92$, 16 items); personality/ideology cluster composite ($\alpha = .92$, 30 items).
\end{tablenotes}
\end{threeparttable}
\end{table}

\begin{table}[htbp]
\caption{Descriptive Statistics for All Primary Measures}
\label{tab:descriptives}
\begin{threeparttable}
\small
\begin{tabular}{lccccc}
\toprule
Variable & $M$ & $SD$ & Mdn & \% Below & \% Above \\
\midrule
Punitiveness (z) & 0.00 & 0.82 & $-$0.01 & 50 & 50 \\
Crime Rates & 5.00 & 1.42 & 5.00 & 19 & 72 \\
Fear & 3.02 & 1.34 & 2.67 & 70 & 21 \\
Hatred & 4.34 & 1.44 & 4.33 & 38 & 54 \\
Anger & 4.84 & 1.51 & 5.00 & 22 & 66 \\
Exclusion & 4.46 & 1.45 & 4.67 & 32 & 59 \\
Degradation & 3.66 & 1.37 & 3.67 & 58 & 33 \\
Suffering & 3.82 & 1.68 & 4.00 & 45 & 43 \\
Prison Violence & 3.04 & 1.38 & 3.00 & 69 & 18 \\
Harsh Conditions & 3.30 & 1.55 & 3.33 & 64 & 28 \\
Revenge & 4.11 & 1.43 & 4.00 & 40 & 49 \\
Hostile Agg.\ (cluster) & 3.77 & 1.21 & 3.84 & 56 & 42 \\
RWA & 3.72 & 1.65 & 3.80 & 51 & 43 \\
SDO & 2.58 & 1.44 & 2.38 & 81 & 15 \\
Vengefulness & 2.97 & 1.41 & 2.80 & 74 & 22 \\
Violence Proneness & 3.29 & 1.41 & 3.25 & 66 & 29 \\
Racial Resentment & 3.52 & 1.76 & 3.50 & 57 & 33 \\
Blood Sports & 3.17 & 1.60 & 3.25 & 64 & 30 \\
\bottomrule
\end{tabular}
\begin{tablenotes}[flushleft]
\small
\item \textit{Note.} $N = 496$. All measures on 1--7 scales except Punitiveness ($z$-scored composite). \% Below/Above = percentage below/above scale midpoint (4.0).
\end{tablenotes}
\end{threeparttable}
\end{table}

\begin{table}[htbp]
\caption{Correlations Between Text Features and Psychological Measures: Testing the Prosocial Facade Hypothesis}
\label{tab:nlp_facade}
\begin{threeparttable}
\small
\begin{tabular}{lccccc}
\toprule
 & \multicolumn{5}{c}{Psychological Measure} \\
\cmidrule(lr){2-6}
Text Feature & Punitiveness & Hostile Agg. & Crime Conc. & Hatred & Revenge \\
\midrule
Prosocial Similarity    & $-.13$** & $-.04$    & .02       & $-.04$    & $-.05$    \\
Dark Similarity         & .03      & $-.01$    & $-.03$    & $-.07$    & .03       \\
Pro--Dark Gap           & $-.17$***& $-.03$    & .06       & .04       & $-.09$*   \\
Dict.\ Prosocial        & $-.10$*  & $-.08$    & $-.11$*   & $-.08$    & $-.06$    \\
Sentiment               & $-.19$***& $-.05$    & $-.08$    & $-.09$*   & $-.02$    \\
\bottomrule
\end{tabular}
\begin{tablenotes}[flushleft]
\small
\item \textit{Note.} $N = 496$. Prosocial and Dark Similarity from Sentence-BERT; Pro--Dark Gap = Prosocial minus Dark similarity; Dict.\ Prosocial = dictionary-coded prosocial proportion; Sentiment = VADER compound score. * $p < .05$, ** $p < .01$, *** $p < .001$.
\end{tablenotes}
\end{threeparttable}
\end{table}

\begin{table}[htbp]
\caption{Language Features by Political Group}
\label{tab:political_moderation}
\begin{threeparttable}
\small
\begin{tabular}{lccc}
\toprule
 & Liberal & Moderate & Conservative \\
 & ($n = 190$) & ($n = 116$) & ($n = 190$) \\
\midrule
$r$(Prosocial Sim., Hostile Agg.) & .03 & $-.05$ & $-.08$ \\
$r$(Dark Sim., Hostile Agg.)      & .05 & $-.11$ & .04    \\
$r$(Sentiment, Punitiveness)      & $-.16$ & $-.30$ & $-.05$ \\
\% Responses Closer to Dark       & 67.9 & 71.6 & 64.7 \\
\bottomrule
\end{tabular}
\begin{tablenotes}[flushleft]
\small
\item \textit{Note.} $N = 496$. \% Closer to Dark = percentage of responses with greater Sentence-BERT cosine similarity to dark prototypes than prosocial prototypes.
\end{tablenotes}
\end{threeparttable}
\end{table}



% === REFERENCES ===
\bibliography{references}

% === APPENDICES / SUPPLEMENTARY ===
\appendix
\section{Supplementary Materials}

\subsection{Full Item Wordings}

All items were rated on a 7-point scale (1 = \textit{Strongly disagree} to 7 = \textit{Strongly agree}) unless otherwise noted. Reverse-scored items are marked with (R).

\subsubsection{Punitiveness Measures}

\textbf{Punish More (2 items).} ``People who commit violent crimes should be punished more harshly than they currently are''; ``The sentences given to violent criminals are too lenient'' (R: ``People who commit violent crimes should be punished \textit{less} harshly than they currently are'').

\textbf{Parsimony (2 items).} ``Courts should impose the minimum sentence necessary to achieve justice'' (R); ``Judges should err on the side of shorter rather than longer sentences when the appropriate punishment is unclear'' (R).

\textbf{Three Strikes (2 items).} ``Three strikes laws (which mandate life sentences for a third felony conviction) are a good idea''; ``Three strikes laws help keep society safe.''

\textbf{LWOP (1 item).} ``Life in prison without the possibility of parole is an appropriate sentence for the most serious crimes.''

\textbf{Death Penalty (1 item).} ``The death penalty is an appropriate punishment for the most serious crimes.''

\textbf{Sentencing Decision.} ``Based on the case you just read, how many years in prison would you sentence Darryl Smith to?'' (0--50 year scale).

\subsubsection{Crime Concerns}

\textbf{Crime Rates (2 items).} ``Crime in America is at an all-time high''; ``Crime rates in the United States have been increasing in recent years.''

\textbf{Fear of Crime (3 items).} ``I am afraid of becoming a victim of a violent crime'' (R: recoded); ``I feel safe walking alone at night in my neighborhood''; ``I worry about being robbed or attacked.''

\subsubsection{Emotional Reactions}

\textbf{Hatred (3 items).} ``I hate people who commit violent crimes''; ``People who commit violent crimes don't deserve to be hated'' (R); ``I feel hatred toward convicted criminals.''

\textbf{Anger (2 items).} ``I feel angry when I hear about violent crimes''; ``Stories about violent crimes make my blood boil.''

\subsubsection{Hostile Aggression}

\textbf{Exclusion (3 items).} Endorsement of civic and social exclusion of offenders (e.g., ``People who commit serious crimes should lose their right to vote'').

\textbf{Degradation (3 items).} Acceptance of shaming and degrading treatment of offenders (e.g., ``It is acceptable for prisons to use humiliating punishments''; R: ``Prisoners should be treated with dignity'').

\textbf{Suffering (2 items).} Endorsement of suffering as appropriate (e.g., ``Criminals should suffer for what they have done'').

\textbf{Prison Violence (2 items).} Tolerance of violence in prisons (R: ``It is the responsibility of the prison system to protect inmates from violence''; R: ``Prison violence is a serious problem that needs to be addressed'').

\textbf{Harsh Conditions (3 items).} Endorsement of harsh incarceration conditions (e.g., ``Prisons should be tough, uncomfortable places''; R: ``Prisoners should have access to educational and recreational programs'').

\textbf{Revenge (3 items).} Endorsement of retaliatory punishment (e.g., ``An eye for an eye is a good principle for the justice system'').

\subsubsection{Personality and Ideology}

\textbf{RWA (5 items).} Adapted from \citet{AltmeyerRWA}. Sample item: ``What our country really needs is a strong, determined leader who will crush evil and take us back to our true path'' (R: ``There is nothing wrong with premarital sexual intercourse'').

\textbf{SDO (8 items).} Adapted from \citet{Pratto1994}. Sample items: ``Some groups of people are simply inferior to other groups''; ``Group equality should be our ideal'' (R).

\textbf{Vengefulness (5 items).} Dispositional tendency to seek revenge. Sample item: ``I tend to hold grudges'' (R: ``I can usually forgive and forget when someone does me wrong'').

\textbf{Violence Proneness (4 items).} Tendency to endorse or engage in violence. Sample item: ``I have threatened people I know.''

\textbf{Racial Resentment (4 items).} Adapted from \citet{HenrySears2002}. Sample items: ``It's really a matter of some people not trying hard enough; if blacks would only try harder they could be just as well off as whites'' (R: ``Generations of slavery and discrimination have created conditions that make it difficult for blacks to work their way out of the lower class'').

\textbf{Blood Sports (4 items).} Frequency of watching violent sports content: boxing, MMA/UFC, professional wrestling, and hunting (rated on 7-point frequency scale from \textit{Never} to \textit{Very often}). Extracted from a broader 9-item media consumption matrix.

\subsubsection{Exploratory Measures}

\textbf{Due Process (3 items).} Tolerance for due process violations (e.g., ``It is more important to convict the guilty than to protect the rights of the accused''; R: ``I would rather see a guilty person go free than an innocent person be convicted'').

\textbf{Uncertain Evidence (4 items).} Willingness to convict on uncertain evidence (e.g., ``If the police arrest someone, that person is probably guilty'').

\subsection{NLP Pipeline Details}

\subsubsection{Dictionary Construction}

Custom dictionaries were constructed for ten justification categories. The prosocial categories were deterrence (e.g., \textit{deter, prevent, discourage, example, lesson}), incapacitation (e.g., \textit{protect, remove, dangerous, threat, isolate}), rehabilitation (e.g., \textit{reform, change, treatment, program, help}), retribution (e.g., \textit{deserve, fair, proportional, fitting, earned}), and norm expression (e.g., \textit{wrong, unacceptable, standard, values, society}). The dark categories were revenge (e.g., \textit{payback, avenge, retaliate, score, taste}), suffering (e.g., \textit{suffer, pain, agony, misery, harsh}), degradation (e.g., \textit{shame, humiliate, strip, debase, degrade}), exclusion (e.g., \textit{banish, exile, outcast, remove, expel}), and victim focus (e.g., \textit{victim, family, closure, justice for, loss}).

\subsubsection{Zero-Shot Classification}

We used the \texttt{facebook/bart-large-mnli} model \citep{lewis2020bart} via the HuggingFace Transformers library. The eight candidate labels were: proportional justice, victim closure, rehabilitation and reform, deterrence and prevention, public safety and protection, societal condemnation (prosocial), and punishment and suffering, revenge and payback (dark). In the multi-label condition, all eight label probabilities were returned for each response. In the forced-choice condition, the single highest-probability label was assigned.

\subsubsection{Sentence-BERT Semantic Similarity}

We used the \texttt{all-MiniLM-L6-v2} model \citep{reimers2019sentence} to encode each response and a set of prototype sentences into dense vector representations. Prototype sentences represented canonical expressions of each justification category. For example, the deterrence prototype was ``This sentence will deter others from committing similar crimes and will serve as an example to the community.'' Cosine similarity was computed between each response and each prototype, and responses were classified by the category with the highest similarity score.

\subsubsection{Prototype Sensitivity Analysis}

To ensure that the dark-closer finding was not an artifact of specific prototype wording, we re-estimated the classification using two alternative prototype sets. The ``formal'' set used academic framings (e.g., ``The incarceration serves a deterrent function by signaling to potential offenders that criminal behavior carries severe consequences''). The ``colloquial'' set used everyday language (e.g., ``Lock him up so other people think twice before doing something like this''). Results were consistent across all three sets: 67.5\% (original), 73.0\% (formal), and 89.7\% (colloquial) of responses were closer to dark prototypes. The retribution classification sensitivity analysis (see above) further confirmed robustness: the dark-closer rate ranged from 67.5\% to 83.5\% depending on whether retribution was prosocial, removed, or reclassified as dark.

\subsubsection{BERTopic Analysis}

Emergent topics were discovered using BERTopic \citep{grootendorst2022bertopic} with UMAP dimensionality reduction and HDBSCAN clustering. The analysis identified three main topic clusters in the response corpus.

\subsection{Additional Analyses}

\subsubsection{Tautology Sensitivity Analysis}

Some hostile aggression constructs---particularly suffering (``Criminals should suffer for what they have done''), harsh conditions (``Prisons should be tough, uncomfortable places''), prison violence tolerance, and degradation---share conceptual territory with punitiveness measures. To ensure that H2 does not depend on this overlap, we conducted sensitivity analyses that progressively removed the most tautology-prone constructs from the hostile aggression composite.

Three levels were tested. (a) \textit{Original} (all 6 constructs): $r_{\text{hostile}} = .59$, $r_{\text{crime}} = .32$, $Z = 6.26$, $p < .001$. (b) \textit{Conservative} (4 constructs: exclusion, degradation, prison violence, revenge): $r = .55$ vs.\ $r = .32$, $Z = 5.07$, $p < .001$. (c) \textit{Strictest} (2 constructs: exclusion and revenge only): $r = .54$ vs.\ $r = .32$, $Z = 5.05$, $p < .001$. Bootstrapped 95\% CIs for the hostile--crime difference excluded zero at all three levels.

At the individual punitiveness measure level, the strictest specification (exclusion $+$ revenge) yielded significant Steiger's $Z$ for five of six measures: Punish More ($Z = 5.90$), Three Strikes ($Z = 3.18$), LWOP ($Z = 2.91$), Death Penalty ($Z = 5.96$), and Parsimony ($Z = 4.60$); the exception was the sentencing decision ($Z = 1.89$, $p = .059$).

We also tested sensitivity to the low-reliability parsimony scale ($\alpha = .48$) by removing it from the punitiveness composite. Results were virtually unchanged ($r = .58$ vs.\ $r = .33$, $Z = 5.87$, $p < .001$). Even in the most conservative specification---only exclusion and revenge, without parsimony---the hostile aggression advantage remained significant ($r = .55$ vs.\ $r = .33$, $Z = 5.02$, $p < .001$).

\subsubsection{TOST Equivalence Testing}

To strengthen the interpretation of null correlations between hostile aggression and text features, we conducted Two One-Sided Tests (TOST) equivalence testing \citep{lakens2017equivalence} with equivalence bounds of $\Delta = \pm .15$ (representing a small effect). Results confirmed formal equivalence to zero for four of five correlations: prosocial similarity ($r = -.04$, $p_{\text{TOST}} = .008$), dark similarity ($r = -.01$, $p_{\text{TOST}} < .001$), the prosocial--dark gap ($r = -.03$, $p_{\text{TOST}} = .005$), and sentiment ($r = -.05$, $p_{\text{TOST}} = .014$). Dictionary-coded prosocial content fell marginally outside the equivalence bounds ($r = -.08$, $p_{\text{TOST}} = .055$). A supplementary crime concerns test confirmed that the prosocial--dark gap was also equivalent to zero for crime concerns ($r = .06$, $p_{\text{TOST}} = .019$).

\subsubsection{Confirmatory Factor Analysis}

We tested the four-cluster structure (hostile aggression, crime concerns, personality, and emotions) using confirmatory factor analysis. The four-factor model yielded $\chi^2(98) = 699.10$, CFI $= .858$, TLI $= .826$, RMSEA $= .111$ [.104, .119], SRMR $= .069$. A five-factor model separating due process from crime concerns provided marginally different fit (CFI $= .858$, TLI $= .826$, RMSEA $= .105$). Both models fell below conventional fit thresholds (CFI $> .90$, RMSEA $< .08$), indicating that the clusters are best understood as conceptual groupings rather than empirically validated latent factors. Standardized factor loadings ranged from .38 to .89, and all were significant at $p < .001$. Factor correlations ranged from $r = .49$ to $r = .81$, consistent with the strong intercorrelations reported in the main text.

\subsubsection{Retribution Classification Sensitivity}

In the primary Sentence-BERT analysis, retribution was classified as prosocial alongside deterrence, incapacitation, and rehabilitation, while revenge, suffering, and exclusion were classified as dark. Because retribution is philosophically ambiguous---sharing conceptual territory with both proportional justice and revenge---we tested two alternative specifications. Removing retribution from the prosocial set (yielding a balanced 3 vs.\ 3 comparison) increased the dark-closer rate to 78.6\%. Moving retribution to the dark side (3 prosocial vs.\ 4 dark) increased it further to 83.5\%. The facade-relevant null correlations were unaffected: hostile aggression remained uncorrelated with both prosocial similarity (all $r < |.04|$, $p > .37$) and the prosocial--dark gap (all $r < |.04|$, $p > .44$) under all three specifications.

\subsubsection{Vignette Stability}

Correlations between punitiveness and the 18 correlate constructs were examined separately within each of the three vignette conditions to assess the stability of findings across crime types. The pattern of results was highly consistent across vignettes, with a mean absolute range of $r = .13$ across the three conditions. No construct showed a qualitatively different pattern in any vignette condition, supporting the generalizability of the findings.

\subsubsection{Political Moderation}

We tested whether political orientation moderated the relationship between hostile aggression and punitiveness, and between hostile aggression and language features. Participants were classified as liberal ($n = 190$), moderate ($n = 116$), or conservative ($n = 190$) based on a single political identification item. Using hierarchical regression with an interaction term (hostile aggression $\times$ political group), none of the interactions were statistically significant (all $p > .24$), indicating that the psychological correlates of punitiveness operate similarly across the political spectrum. The dark-closer percentage was similar across groups: 67.9\% (liberal), 71.6\% (moderate), and 64.7\% (conservative).

\subsubsection{Cross-Method Convergence}

Pairwise agreement rates across the four NLP classification methods ranged from 50.1\% to 90.5\% for raw agreement, with Cohen's $\kappa$ values ranging from .03 to .66. The two BART-based methods (multi-label and forced-choice) showed the highest agreement (90.5\%, $\kappa = .66$), as expected given their shared underlying model. Agreement between BART-based methods and the dictionary or Sentence-BERT similarity methods was more modest ($\kappa = .03$--.11), reflecting the fundamentally different principles of operation---keyword matching, zero-shot entailment, and dense semantic similarity capture different aspects of meaning. Importantly, all methods converged on the central finding: prosocial categories dominated the surface-level classifications (80--88\% prosocial by three methods), while Sentence-BERT semantic similarity revealed that 67.5\% of responses were closer to dark prototypes at the deeper semantic level.

\subsubsection{Item-Level Correlations}

Item-level correlations between individual survey items and the punitiveness composite are available in the online supplementary materials at \url{https://dgk-law-and-cognition-lab.github.io/Punishment-Punitiveness/}. All 56 items showed the expected direction of correlation, and 52 of 56 were statistically significant at $p < .05$.

\subsection{Supplementary Figures}

The following supplementary figures are available at the online materials page:

\textbf{Figure S1.} Full intercorrelation heatmap showing all 153 pairwise correlations among the 18 correlate constructs.

\textbf{Figure S2.} UMAP visualization of the response corpus. Four-panel figure showing (a) responses colored by punitiveness, (b) responses colored by prosocial--dark language gap, (c) hostile aggression vs.\ prosocial language similarity, and (d) distribution of facade residuals.

\textbf{Figure S3.} Word clouds for the top and bottom tertiles of hostile aggression, illustrating the similarity of language used across groups.

\textbf{Figure S4.} Prototype sensitivity analysis: distributions of prosocial--dark similarity difference under four prototype specifications (original, formal, colloquial, and data-driven).

\textbf{Figure S5.} NLP feature correlation heatmap showing relationships among all text-based features and psychological measures.

\textbf{Figure S6.} TOST equivalence testing visualization. All five hostile aggression $\times$ text feature correlations and their 95\% CIs, plotted against equivalence bounds of $\pm .15$. All CIs fall within the green equivalence region.


\end{document}
