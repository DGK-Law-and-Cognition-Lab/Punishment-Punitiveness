\section{Results}

\subsection{Preliminary Analyses}

\subsubsection{Sample Characteristics}

The final sample comprised 496 U.S. adults. Participants were randomly assigned to one of three vignettes: Stranger Felony-Murder ($n = 168$), Domestic Violence ($n = 176$), and Organized Crime ($n = 152$). Political orientation was assessed on a 7-point scale and categorized into Liberal ($n = 190$), Moderate ($n = 116$), and Conservative ($n = 190$).

\subsubsection{Descriptive Statistics}

The punitiveness composite ($M = 0.00$, $SD = 0.82$) was centered at zero by construction. Among the correlate measures, the highest means were observed for anger toward criminals ($M = 4.84$, $SD = 1.51$), perceived crime rates ($M = 5.00$, $SD = 1.42$), and exclusion ($M = 4.46$, $SD = 1.45$). The lowest means were for social dominance orientation ($M = 2.58$, $SD = 1.44$), vengefulness ($M = 2.97$, $SD = 1.41$), and fear of crime ($M = 3.02$, $SD = 1.34$). Notably, the hostile aggression composite ($M = 3.77$, $SD = 1.21$) fell near the scale midpoint, with 42\% of participants scoring above the midpoint. Full descriptive statistics are reported in Table~\ref{tab:descriptives}.

\subsubsection{Reliability}

All primary composites demonstrated acceptable to excellent internal consistency. The punitiveness composite ($\alpha = .84$), the hostile aggression cluster ($\alpha = .92$), and the personality cluster ($\alpha = .92$) all exceeded $\alpha = .80$. Two individual constructs---parsimony ($\alpha = .48$) and prison violence tolerance ($\alpha = .56$)---fell below conventional thresholds and are interpreted with caution. Full reliability estimates are reported in Table~\ref{tab:reliability}.

\subsection{Hypothesis 1: Punitiveness Correlates With All Measures}

Hypothesis 1 predicted that punitiveness would be positively correlated with all correlate measures. This prediction was strongly supported: 16 of 18 constructs were significantly and positively correlated with punitiveness (all $p < .01$ after FDR correction). The two exceptions were vengefulness ($r = .12$, $p = .074$) and blood sports viewership ($r = .11$, $p = .080$), which were positive but fell short of significance. The full results are presented in Table~\ref{tab:H1correlations} and Figure~\ref{fig:heatmap_construct}.

The strongest correlates of punitiveness were: endorsement of harsh prison conditions ($r = .56$), social exclusion ($r = .54$), hatred of criminals ($r = .53$), infliction of suffering ($r = .52$), right-wing authoritarianism ($r = .50$), and racial resentment ($r = .46$). By contrast, the crime concern measures---perceived crime rates ($r = .40$) and fear of crime ($r = .18$)---correlated substantially less strongly with punitiveness. The pattern is clear: measures tapping hostile, aggressive, and ideological dispositions correlate with punitiveness at least as strongly as---and typically more strongly than---measures of legitimate concern about crime.

\subsection{Hypothesis 2: Hostile Aggression Outpredicts Crime Concerns}

Hypothesis 2 predicted that the hostile aggression cluster would correlate more strongly with punitiveness than the crime concerns cluster. This was tested using Steiger's $Z$-test for dependent correlations \citep{steiger1980tests}. The hostile aggression composite correlated with punitiveness at $r = .59$, whereas the crime concerns composite correlated at $r = .32$. This difference was highly significant ($Z = 6.26$, $p < .001$).\footnote{When punitiveness is measured using the 8-item attitude composite alone (excluding the sentencing decision), the hostile aggression correlation increases to $r = .71$ ($r_{\text{crime}} = .36$), producing an even larger difference.}

We further tested this hypothesis for each individual punitiveness measure to ensure robustness. As shown in Table~\ref{tab:steiger}, hostile aggression significantly outpredicted crime concerns for five of six punitiveness measures: Punish More ($\Delta r = .34$, $Z = 8.23$, $p < .001$), Parsimony ($\Delta r = .31$, $Z = 6.70$, $p < .001$), Death Penalty ($\Delta r = .32$, $Z = 7.23$, $p < .001$), Three Strikes ($\Delta r = .22$, $Z = 5.06$, $p < .001$), and LWOP ($\Delta r = .15$, $Z = 3.27$, $p = .001$). The one exception was the sentencing decision ($\Delta r = .08$, $Z = 1.60$, $p = .11$), for which the difference was in the predicted direction but not statistically significant.

Bootstrapped 95\% confidence intervals (10,000 iterations) further confirmed that the advantage of hostile aggression over crime concerns was robust (Figure~\ref{fig:bootstrap}). At the cluster level, the observed difference of $r = .264$ had a bootstrap CI of $[.174, .367]$, excluding zero. The same held for the emotions cluster (difference $= .213$, CI $[.115, .321]$) and the personality cluster (difference $= .153$, CI $[.057, .259]$).

\subsubsection{Tautology Sensitivity Analysis}

A potential concern is that some hostile aggression constructs---particularly suffering, harsh conditions, prison violence, and degradation---may overlap conceptually with punitiveness, inflating the hostile aggression correlation through construct overlap rather than genuine predictive superiority. We addressed this through a series of sensitivity analyses in which we progressively removed the most tautology-prone constructs from the hostile aggression composite. In the conservative test (retaining only exclusion, degradation, prison violence, and revenge), the hostile aggression advantage held: $r = .55$ vs.\ $r = .32$, $Z = 5.07$, $p < .001$. In the strictest test (retaining only exclusion and revenge---the two constructs with no plausible item overlap with punitiveness), the advantage remained: $r = .54$ vs.\ $r = .32$, $Z = 5.05$, $p < .001$. The strictest test held for five of six individual punitiveness measures (all $Z > 2.91$, $p < .004$; the exception was again the sentencing decision). We further tested robustness to the lowest-reliability component of the punitiveness composite by dropping parsimony ($\alpha = .48$); results were virtually unchanged ($r = .58$ vs.\ $r = .33$, $Z = 5.87$, $p < .001$). Even in the most conservative specification---only two hostile aggression constructs and no parsimony items---the hostile aggression advantage was significant ($r = .55$ vs.\ $r = .33$, $Z = 5.02$, $p < .001$). Full details are reported in the Supplementary Materials.

\subsection{Hypothesis 3: Cohesive Punitiveness Mindset}

Hypothesis 3 predicted that the correlate measures would be positively intercorrelated, forming a cohesive psychological mindset. This prediction was strongly supported. All 153 pairwise correlations among the 18 constructs were positive, ranging from $r = .02$ (fear--racial resentment) to $r = .72$ (hatred--anger), and the vast majority were statistically significant. The full intercorrelation matrix is presented in the supplementary materials (Table~S1).

Several notable patterns emerged. The hostile aggression constructs formed a tightly interrelated cluster, with intercorrelations ranging from $r = .38$ (prison violence--revenge) to $r = .71$ (exclusion--degradation). Emotional reactions (hatred and anger) were strongly intercorrelated ($r = .72$) and showed robust associations with hostile aggression constructs ($r$s = .15--.64, with the lower values involving prison violence tolerance, the least reliable construct in the cluster). Right-wing authoritarianism and social dominance orientation, while moderately correlated with each other ($r = .43$), showed differential patterns: RWA correlated more strongly with crime-related constructs (crime rates: $r = .44$; anger: $r = .49$) than did SDO ($r$s = .20 and .24, respectively), whereas SDO correlated more strongly with racial resentment ($r = .62$ vs.\ $r = .53$).

The two constructs that fell outside the mindset---vengefulness and blood sports---provide an informative contrast. Vengefulness was weakly correlated with punitiveness ($r = .12$) but moderately correlated with revenge ($r = .46$), degradation ($r = .37$), and suffering ($r = .40$). This suggests that \textit{dispositional} vengefulness (the tendency to seek revenge in everyday life) is dissociable from the \textit{domain-specific} endorsement of harsh criminal punishment, even though both share variance with aggressive treatment of offenders. Blood sports viewership was weakly associated with the entire mindset, suggesting that preference for violent entertainment operates largely independently of the punitiveness ecosystem.

\subsection{Vignette Stability}

Because participants were randomly assigned to one of three vignettes, we examined whether the correlational patterns were stable across crime types. The punitiveness--correlate correlations were highly stable: across the 18 constructs, the range of correlations across vignettes was small (median range $= .14$), and no construct reversed direction across conditions. The hostile aggression advantage over crime concerns was replicated within each vignette condition. This stability indicates that the punitiveness mindset is not an artifact of the particular crime scenario presented to participants.
