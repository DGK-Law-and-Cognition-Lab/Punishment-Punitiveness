% === INTRODUCTION ===
% No heading needed per APA 7 - text begins directly after abstract

Criminal punishment has always been in tension with liberal democratic values and enlightened human sensibilities \citep{Allen1999}. As Jeremy Bentham (1789) reminds us: ``all punishment is mischief: all punishment in itself is evil'' (ch.~13, para.~2). For centuries, this tension has precipitated the inquiry ``why do we punish?'' \citep{Allen1999, Tonry2011}, an endeavor that has sought to couch punishment in a just framework. Classically, the well-honed justifications of punishment consist of retribution, deterrence, incapacitation, norm expression, and rehabilitation \citep[see][]{Norrie2014, Tonry2011}. These justifications dominate our legal and political discourse. They serve as the gateway to legal education \citep{Ristroph2006}, and they figure prominently in courtroom rhetoric, judicial opinions, legislative debates, and political campaigning. We rely on them also when we speak of bringing offenders to justice, getting them off the street, locking them up, and sending them a message.

This paper does not attempt to examine the philosophical soundness of these justifications, nor to explore how they fit with one another (they do not). Rather, we focus on the core tenet that they are said to serve. The justifications claim great fidelity to the pursuit of desirable and race-neutral societal goals: restoring justice, reducing crime, reinforcing social norms, and rehabilitating people who have broken the law---goals that are quintessentially prosocial.

While grounded in moral thinking \citep{Allen1999, Tonry2011}, the prosocial tenet enjoys empirical backing. Experimental economists and evolutionary psychologists highlight the altruistic nature of punishment. Punishment is said to have enabled our predecessors to contain important collective problems such as free-loading, mate poaching, and acts of violence. Punishing such behaviors has served to facilitate non-kin cooperation and promote societal coexistence \citep{BowlesGintis2011, PriceCosmidesTooby2002, ShinadaYamagishi2007, FehrFischbacher2003, FehrFischbacher2004, FehrGachter2000, FehrGachter2002}.

It must not be overlooked that we are conducting this debate in the shadow of a fifty-year punitive regime that is excessive in its scope \citep{NRC2014, Garland2020}, harshness \citep{Gottschalk2016, JSimon2016}, and disparate racial impact \citep{Armour2020, Alexander2010, Western2006}. The question that animates this exploration is how we get from the avowed prosociality of punishment to mass incarceration, and why we maintain this regime while knowing that it causes more harm than good \citep[see][]{NRC2014, Tonry2018}. The answer, we maintain, cannot be chalked up solely to ignorance or institutional inertia. Rather, it may well lie in the composition and force of the psychological motivations that fuel the punitive mindset.

\subsection{The Punitive Mindset}

The psychology of punitiveness is complex, touching on beliefs, emotions, attitudes, personality traits, worldviews, and ideologies. A considerable body of research has explored the correlates of punitive attitudes, revealing a picture that sits uneasily alongside the prosocial premise.

Punitiveness is shaped, in part, by people's perceptions and fears about crime. Belief in high crime rates is associated with greater punitiveness \citep{Hough2005, ApplegateEtAl2002, SprucerEtAl2015}, even though such beliefs are frequently inflated relative to actual crime trends \citep{QuillianPager2010, Pfaff2017}. Fear of becoming a victim also correlates with punitive attitudes \citep{Hough2005}, though this relationship is more modest than is commonly supposed \citep{Gerber2021, Pickett2019}. These crime-related concerns represent the most intuitively prosocial correlates of punitiveness---people who perceive greater threats might reasonably desire stronger responses.

Emotions, however, tell a more complicated story. Anger and outrage toward offenders are potent predictors of punitive responses \citep{Goldberg1999, Stalans1993, Carlsmith2008}, and emotional reactions can override more deliberative considerations \citep{Lerner2000}. Hatred of criminals is closely associated with punitiveness \citep{DSimon2023}, and people endorse harsher punishments when they experience strong negative affect toward a transgressor \citep{Roseman1994, Weiner1995}. These emotional reactions are not necessarily inimical to prosociality---anger at injustice can motivate constructive action---but they move the motivational story beyond the cool rationality implied by the deterrence and incapacitation frameworks.

Perhaps most troubling for the prosocial account, punitiveness appears to be closely intertwined with endorsement of hostile and aggressive treatment of offenders. Punitive attitudes correlate with acceptance of degrading prison conditions \citep{DSimon2023}, tolerance of prison violence, and the desire to inflict suffering on offenders beyond what is required for any utilitarian purpose. People also seem to derive satisfaction from incidental suffering that befalls rule violators \citep{FincherTetlock2015, GillCerce2021}. Punitiveness is closely intertwined with revenge \citep{Jackson2019, GerberJackson2013, McKeeFeather2008}, which is predicted by the arousal of anger \citep{Dewall2007, Chester2016} and is associated with aggression and violence \citep{DewallChester2021}. Revenge often results in disproportionate punishments that exceed the severity of the transgression \citep{Ho2002}. Critically, experimental research has consistently shown that when people's punishment decisions are examined closely, they are better explained by retribution and revenge than by the utilitarian justifications of deterrence or incapacitation that people tend to endorse verbally \citep{Carlsmith2006, Carlsmith2008, CarlsmithDarley2008, AharoniFridlund2012}.

Punitive attitudes are also deeply embedded in broader personality and ideological orientations. Right-wing authoritarianism (RWA) is among the most robust predictors of punitiveness \citep{Gerber2021, GerberJackson2013, AltmeyerRWA}, and social dominance orientation (SDO) is likewise associated with punitive attitudes \citep{Sidanius2004, Pratto1994}. Violence proneness, racial resentment, and vengefulness have all been linked to harsher punishment preferences \citep{KindaGoldberg2020, Unnever2010a, Unnever2010b}. Taken together, these personality and ideological factors paint a portrait of punitiveness that is deeply interwoven with authoritarian, hierarchical, and retaliatory dispositions.

This partial review reveals that punitiveness is a highly complex and fraught psychological construct. Yet this literature is dispersed across a large number of studies, each of which tests a small number of constructs and thus covers limited slivers of the problem space \citep[for partial exceptions, see][]{KingMaruna2009, GerberJackson2013, Okimoto2012}. Consequently, the field has lacked a macro-level view that gauges the relative strength of the various correlate constructs and examines their interrelationships. Additionally, there is no standard operationalization of the focal punitiveness construct \citep[see][]{Gerber2021}: some studies base it on penological attitudes \citep[e.g.,][]{Cullen1985}, others solicit sentencing decisions in response to vignettes \citep[e.g.,][]{Carlsmith2008}, and yet others tap support for specific sentencing policies \citep[e.g.,][]{TylerBoeckmann1997}.

\subsection{The Problem of Self-Report}

The research reviewed above relies almost exclusively on what people report about themselves on structured questionnaires. This is a problem for studying punitiveness, because the prosocial justifications of punishment are deeply embedded in public discourse. When asked \textit{why} they endorse harsh sentences, people reliably invoke deterrence, public safety, and incapacitation \citep{Carlsmith2008, Roberts2003}. Yet a separate line of experimental research demonstrates that these stated justifications often do not match people's actual punishment behavior: people endorse deterrence in principle but punish according to retributive severity in practice \citep{Carlsmith2006, CarlsmithDarley2008}.

This discrepancy raises a critical question: when people justify their punishment preferences in their own words, are they expressing their genuine motivations, or are they drawing on a culturally available script? If the prosocial language of deterrence and rehabilitation functions as a rhetorical veneer---a \textit{prosocial facade}---then examining people's spontaneous justifications with computational text analysis may reveal the gap between what they say and what drives them.

\subsection{The Present Study}

The present study addresses these limitations with two complementary approaches. First, we conduct a comprehensive correlational investigation of the punitiveness ecosystem, measuring how eighteen psychological constructs---organized into four thematic clusters (crime concerns, emotional reactions, hostile aggression endorsement, and personality/ideological factors)---relate to punitiveness and to one another. This design enables us to examine the field at both a macro-level cluster view and a detailed construct-level view, providing the kind of integrative framework that has been lacking in the literature. We operationalize punitiveness using a multi-method composite that combines penological attitudes, policy preferences, and a sentencing decision across three criminal vignettes, addressing the longstanding operationalization problem.

Second, and centrally, we extend this correlational approach by collecting open-ended sentencing justifications and subjecting them to a multi-method natural language processing (NLP) pipeline. Following their sentencing decision, participants were asked to explain their reasoning in their own words. We then applied four independent computational text analysis methods---dictionary-based coding, zero-shot classification with large language models, forced-choice classification, and Sentence-BERT semantic similarity---to characterize these justifications. This approach allows us to address the prosocial facade hypothesis directly: we can test whether participants' spontaneous language aligns with the prosocial justifications or with the darker psychological factors that correlate with their attitudes.

This study is, to our knowledge, the first to combine a comprehensive psychological profiling of punitiveness with computational text analysis of punishment justifications. The NLP approach offers several advantages over prior methods. Unlike experimental paradigms that pit deterrence against retribution in artificial trade-off scenarios \citep{Carlsmith2006}, text analysis examines the language people naturally produce. Unlike human coding, computational methods are scalable, reproducible, and less susceptible to coder bias. And unlike prior qualitative studies of punishment language, multi-method NLP can quantify the semantic content of justifications and correlate it directly with the psychological profiles established through the quantitative measures.

We preregistered the following primary hypotheses (\url{https://osf.io/kr7y2/}). \textbf{Hypothesis 1} predicted that punitiveness would be positively correlated with all correlate measures. \textbf{Hypothesis 2} predicted that the hostile aggression cluster would show stronger correlations with punitiveness than the crime concerns cluster. \textbf{Hypothesis 3} predicted that most correlate measures would be positively intercorrelated, suggesting a cohesive psychological ``punitiveness mindset.''

For the NLP analyses, we tested two additional predictions. \textbf{Hypothesis 4} predicted that participants would predominantly use prosocial language (deterrence, public safety, rehabilitation) in their sentencing justifications, even when their psychological profiles suggest hostile motivations. \textbf{Hypothesis 5} predicted that the semantic content of justifications would be more closely aligned with revenge and suffering themes than with deterrence and rehabilitation, despite surface-level prosocial framing.
