\section{Discussion}

This study set out to probe the prosocial premise of criminal punishment---the bedrock assumption that punishment serves desirable societal goals such as deterrence, public safety, and rehabilitation. We pursued this question through two complementary lenses: a comprehensive correlational mapping of the psychological factors associated with punitiveness, and a multi-method computational text analysis of the language people use to justify their punishment preferences. Together, these approaches paint a picture that is difficult to reconcile with the prosocial narrative.

\subsection{The Punitiveness Mindset}

The correlational findings reveal that punitiveness is embedded in a dense, cohesive psychological ecosystem. Virtually all of the constructs we measured---spanning beliefs, emotions, aggressive dispositions, personality traits, and political ideologies---correlated positively with punitiveness and with one another. This is not a scattershot collection of weak associations. The hostile aggression cluster---comprising hatred of criminals, endorsement of degradation, desire for suffering, tolerance of prison violence, support for harsh conditions, and revenge---correlated with punitiveness at $r = .59$, dramatically exceeding the $r = .32$ correlation observed for crime concerns. This pattern held across five of six individual punitiveness measures, was confirmed by bootstrapped confidence intervals, and survived a series of tautology sensitivity tests that progressively stripped the hostile aggression cluster down to its two most conceptually distinct components (exclusion and revenge; $r = .54$ vs.\ $r = .32$, $Z = 5.05$, $p < .001$).

These findings converge with and extend prior research linking punitiveness to authoritarian dispositions \citep{GerberJackson2013}, negative emotions \citep{Goldberg1999}, and revenge motives \citep{Ho2002, Jackson2019}. What the present study adds is the \textit{relative magnitude} of these associations: when hostile aggression, crime concerns, emotional reactions, and personality factors are examined side by side, it is the aggressive, retaliatory, and hierarchical dispositions that show the strongest ties to punitive attitudes---not the concern about crime that would be expected under the prosocial account.

\subsection{What People Say vs.\ What Drives Them}

The NLP findings introduce a new dimension to this picture. When asked to explain their sentencing decisions in their own words, participants overwhelmingly invoked prosocial language. Deterrence, public safety, and proportional justice were the dominant themes across all classification methods. At face value, this would seem to affirm the prosocial premise.

But the semantic analysis told a different story. Despite the prosocial framing, 67.5\% of justifications were semantically closer to revenge, suffering, and exclusion themes than to deterrence, rehabilitation, and incapacitation---a finding robust across four independent classification methods, three alternative sets of semantic prototypes, and alternative classifications of the philosophically ambiguous category of retribution. People said ``deterrence'' and ``public safety,'' but the substance of what they wrote resonated more with revenge and suffering.

\subsection{The Prosocial Facade as Cultural Default}

The most important finding is the nature of this mismatch. We hypothesized that it might reflect an individual-level facade: that people high in hostile aggression would strategically adopt prosocial language to mask their darker motivations. This hypothesis was not supported. Hostile aggression was completely uncorrelated with all text features---prosocial language, dark language, the prosocial--dark gap, and sentiment. People high and low in hostile aggression produced virtually identical justifications.

Nor did the mismatch reflect genuine prosocial sincerity: crime concerns were equally uncorrelated with the prosocial--dark gap. Instead, the evidence converged on what we term a \textit{cultural default}---a shared, collectively maintained rhetorical script that people draw on when justifying punishment, regardless of their actual psychological profile. The prosocial language of deterrence and public safety is so deeply embedded in legal discourse, political rhetoric, and media coverage that it functions as the default vocabulary for discussing punishment. Everyone speaks the language of prosociality, but underneath, the motivational substrate varies dramatically.

The political ideology analysis reinforced this interpretation. Conservatives were substantially more punitive and more hostile than liberals, yet their sentencing justifications were linguistically indistinguishable. The cultural default held equally across the political spectrum: liberals, moderates, and conservatives all drew on the same prosocial script, and political orientation did not moderate the relationship between psychological profiles and language use.

\subsection{Implications}

These findings have several implications for how we think about criminal punishment.

First, they challenge the taken-for-granted prosocial framing of punishment. The justifications that dominate our legal and political discourse---deterrence, incapacitation, public safety---may not describe the actual psychological forces that drive punishment policy. The fact that hostile aggression, not crime concern, is the dominant correlate of punitiveness suggests that much of what fuels public demand for harsh sentences has little to do with making society safer and much to do with satisfying retaliatory impulses. The fact that these impulses are dressed in prosocial language makes them harder to identify, scrutinize, and resist.

Second, the cultural default finding suggests that the problem is not one of individual deception. People are not consciously lying when they cite deterrence as their rationale. Rather, they are drawing on a culturally available explanatory framework that has become so pervasive that it functions as the natural way to talk about punishment. This has parallels to what \citet{Wilson2004} has described as the ``adaptive unconscious''---the gap between the reasons we give for our behavior and the actual causes of that behavior. In the domain of punishment, the prosocial justifications may function as what \citet{Haidt2001} calls post-hoc rationalizations: morally satisfying accounts that are constructed after the punitive judgment has already been made on more visceral grounds.

Third, the incremental validity findings suggest that text analysis captures something about punitiveness that standard psychological batteries miss. Open-ended language predicted punitiveness above and beyond hostile aggression, emotions, personality, and crime concerns ($\Delta R^2 = .055$), and it was an even stronger predictor of actual sentencing decisions ($\Delta R^2 = .141$). This points to the potential of computational text analysis as a complementary tool in the study of legal attitudes, capable of tapping dimensions of reasoning that structured scales cannot reach.

Fourth, these findings are relevant to the ongoing debate about mass incarceration in the United States. If punitive policy is driven in substantial part by hostile aggression rather than legitimate crime concerns, and if this fact is obscured by a prosocial rhetorical facade, then the usual policy debates---which take the prosocial goals at face value and argue about whether punishment ``works'' as deterrence or incapacitation---are missing a significant part of the picture. A more honest discourse would acknowledge the role of retaliatory and aggressive motivations and subject them to the same scrutiny we apply to the official justifications \citep[cf.][]{NRC2014, Tonry2018, Garland2020}.

\subsection{Limitations and Future Directions}

Several limitations should be noted. First, the study is cross-sectional and correlational; we cannot establish causal relationships between psychological dispositions and punitiveness. The strong intercorrelations among constructs suggest a common underlying factor, but the causal architecture of this mindset remains to be determined through experimental or longitudinal designs.

Second, the theoretical clustering of constructs into hostile aggression, crime concerns, emotions, and personality was guided by face validity and prior literature rather than confirmed by the data's factor structure. Confirmatory factor analysis of our four-cluster model yielded fit indices below conventional thresholds (CFI $= .86$, RMSEA $= .11$), indicating that the clusters should be understood as conceptual organizers rather than empirically validated latent factors. Relatedly, some hostile aggression constructs (particularly suffering, harsh conditions, and prison violence) share conceptual territory with punitiveness itself, raising a tautology concern. However, sensitivity analyses demonstrated that the hostile aggression advantage over crime concerns survived even when the cluster was stripped to its two most conceptually distinct constructs---exclusion and revenge ($r = .54$ vs.\ $r = .32$, $Z = 5.05$, $p < .001$)---and when parsimony items were simultaneously removed from the punitiveness composite. The substantive conclusion does not depend on the potentially tautological elements.

Third, our sample was recruited through Prolific using nationally representative quotas on age, sex, and ethnicity. Although this approach yields samples that approximate U.S. census demographics more closely than convenience sampling, online samples may still differ from the broader population in ways not captured by quota variables \citep{ChandlerShapiro2016}. The generalizability of our findings to other national contexts or to populations reached through probability sampling requires further investigation.

Fourth, the NLP methods, while multi-method and validated, are imperfect. The semantic similarity approach depends on the prototype sentences used; although sensitivity analyses showed robustness across three prototype sets and alternative classifications of retribution (which strengthened rather than weakened the finding), the specific quantitative estimates (e.g., 67.5\% dark-closer) should be interpreted with appropriate caution. Future research might employ additional text analysis methods, including fine-tuned classification models trained on domain-specific labeled data.

Fifth, our open-ended prompt asked participants to explain their sentencing decision, which may have encouraged post-hoc rationalization. Future studies could elicit justifications before the sentencing decision, or use think-aloud protocols, to examine whether the cultural default operates even when participants are not constructing retrospective explanations.

Sixth, we measured punitiveness in the context of three criminal vignettes involving second-degree murder. The generalizability to other crime types, other legal contexts, or real-world sentencing decisions remains an open question, though the stability of our findings across the three vignettes is encouraging.

\subsection{Conclusion}

Is criminal punishment prosocial? Our findings suggest that the answer is far more complicated than the official discourse implies. The psychological forces most strongly associated with punitiveness---hatred, revenge, degradation, suffering, authoritarianism---are orthogonal to the prosocial goals of deterrence, public safety, and rehabilitation. When people explain their punishment preferences in their own words, they reliably invoke those prosocial goals. But computational analysis of their language reveals that the substance of what they say is more aligned with revenge and suffering than with deterrence and reform. This prosocial framing is not individual-level deception; it is a collective cultural default, a shared rhetorical resource that obscures the darker motivational substrate of punishment.

Coming to terms with this complexity might lead us to reconsider how we think about, talk about, and practice criminal punishment. At a minimum, these findings call for a discourse that demystifies the prevailing prosocial rhetoric and introduces a frank recognition that we punish, in part, to satisfy our own---often unsavory---psychological needs.
